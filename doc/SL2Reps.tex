% generated by GAPDoc2LaTeX from XML source (Frank Luebeck)
\documentclass[a4paper,11pt]{report}

\usepackage[top=37mm,bottom=37mm,left=27mm,right=27mm]{geometry}
\sloppy
\pagestyle{myheadings}
\usepackage{amssymb}
\usepackage[utf8]{inputenc}
\usepackage{makeidx}
\makeindex
\usepackage{color}
\definecolor{FireBrick}{rgb}{0.5812,0.0074,0.0083}
\definecolor{RoyalBlue}{rgb}{0.0236,0.0894,0.6179}
\definecolor{RoyalGreen}{rgb}{0.0236,0.6179,0.0894}
\definecolor{RoyalRed}{rgb}{0.6179,0.0236,0.0894}
\definecolor{LightBlue}{rgb}{0.8544,0.9511,1.0000}
\definecolor{Black}{rgb}{0.0,0.0,0.0}

\definecolor{linkColor}{rgb}{0.0,0.0,0.554}
\definecolor{citeColor}{rgb}{0.0,0.0,0.554}
\definecolor{fileColor}{rgb}{0.0,0.0,0.554}
\definecolor{urlColor}{rgb}{0.0,0.0,0.554}
\definecolor{promptColor}{rgb}{0.0,0.0,0.589}
\definecolor{brkpromptColor}{rgb}{0.589,0.0,0.0}
\definecolor{gapinputColor}{rgb}{0.589,0.0,0.0}
\definecolor{gapoutputColor}{rgb}{0.0,0.0,0.0}

%%  for a long time these were red and blue by default,
%%  now black, but keep variables to overwrite
\definecolor{FuncColor}{rgb}{0.0,0.0,0.0}
%% strange name because of pdflatex bug:
\definecolor{Chapter }{rgb}{0.0,0.0,0.0}
\definecolor{DarkOlive}{rgb}{0.1047,0.2412,0.0064}


\usepackage{fancyvrb}

\usepackage{mathptmx,helvet}
\usepackage[T1]{fontenc}
\usepackage{textcomp}


\usepackage[
            pdftex=true,
            bookmarks=true,        
            a4paper=true,
            pdftitle={Written with GAPDoc},
            pdfcreator={LaTeX with hyperref package / GAPDoc},
            colorlinks=true,
            backref=page,
            breaklinks=true,
            linkcolor=linkColor,
            citecolor=citeColor,
            filecolor=fileColor,
            urlcolor=urlColor,
            pdfpagemode={UseNone}, 
           ]{hyperref}

\newcommand{\maintitlesize}{\fontsize{50}{55}\selectfont}

% write page numbers to a .pnr log file for online help
\newwrite\pagenrlog
\immediate\openout\pagenrlog =\jobname.pnr
\immediate\write\pagenrlog{PAGENRS := [}
\newcommand{\logpage}[1]{\protect\write\pagenrlog{#1, \thepage,}}
%% were never documented, give conflicts with some additional packages

\newcommand{\GAP}{\textsf{GAP}}

%% nicer description environments, allows long labels
\usepackage{enumitem}
\setdescription{style=nextline}

%% depth of toc
\setcounter{tocdepth}{1}





%% command for ColorPrompt style examples
\newcommand{\gapprompt}[1]{\color{promptColor}{\bfseries #1}}
\newcommand{\gapbrkprompt}[1]{\color{brkpromptColor}{\bfseries #1}}
\newcommand{\gapinput}[1]{\color{gapinputColor}{#1}}


\begin{document}

\logpage{[ 0, 0, 0 ]}
\begin{titlepage}
\mbox{}\vfill

\begin{center}{\maintitlesize \textbf{ SL2Reps \mbox{}}}\\
\vfill

\hypersetup{pdftitle= SL2Reps }
\markright{\scriptsize \mbox{}\hfill  SL2Reps  \hfill\mbox{}}
{\Huge \textbf{ Constructs representations of SL2(Z). \mbox{}}}\\
\vfill

{\Huge  0.1 \mbox{}}\\[1cm]
{ 24 September 2021 \mbox{}}\\[1cm]
\mbox{}\\[2cm]
{\Large \textbf{ Siu-Hung Ng\\
  \mbox{}}}\\
{\Large \textbf{ Yilong Wang\\
  \mbox{}}}\\
{\Large \textbf{ Samuel Wilson\\
  \mbox{}}}\\
\hypersetup{pdfauthor= Siu-Hung Ng\\
  ;  Yilong Wang\\
  ;  Samuel Wilson\\
  }
\end{center}\vfill

\mbox{}\\
{\mbox{}\\
\small \noindent \textbf{ Siu-Hung Ng\\
  }  Email: \href{mailto://rng@math.lsu.edu} {\texttt{rng@math.lsu.edu}}}\\
{\mbox{}\\
\small \noindent \textbf{ Yilong Wang\\
  }  Email: \href{mailto://wyl@bimsa.cn} {\texttt{wyl@bimsa.cn}}}\\
{\mbox{}\\
\small \noindent \textbf{ Samuel Wilson\\
  }  Email: \href{mailto://swil311@lsu.edu} {\texttt{swil311@lsu.edu}}}\\
\end{titlepage}

\newpage\setcounter{page}{2}
\newpage

\def\contentsname{Contents\logpage{[ 0, 0, 1 ]}}

\tableofcontents
\newpage

     
\chapter{\textcolor{Chapter }{Introduction}}\label{Chapter_Introduction}
\logpage{[ 1, 0, 0 ]}
\hyperdef{L}{X7DFB63A97E67C0A1}{}
{
  

 This package, \texttt{SL2Reps}, provides methods for constructing and testing matrix presentations of the
representations of $\mathrm{SL}_2(\mathbb{Z})$ whose kernels are congruence subgroups of $\mathrm{SL}_2(\mathbb{Z})$. 

 Irreducible representations of prime-power level are constructed individually
by using the Weil representations of quadratic modules, and from these a list
of all representations of a given degree or level can be produced. The format
is designed for the study of modular tensor categories in particular. 

 
\section{\textcolor{Chapter }{Installation}}\label{Chapter_Introduction_Section_Installation}
\logpage{[ 1, 1, 0 ]}
\hyperdef{L}{X8360C04082558A12}{}
{
  

 To install \texttt{SL2Reps}, first download it from \texttt{TODO}, then place it in the \texttt{pkg} subdirectory of your GAP installation (or in the \texttt{pkg} subdirectory of any other GAP root directory, for example one added with the \texttt{-l} argument). 

 \texttt{SL2Reps} is then loaded with the GAP command 

 \texttt{gap{\textgreater} LoadPackage( "SL2Reps" );} 

 }

 
\section{\textcolor{Chapter }{Usage}}\label{Chapter_Introduction_Section_Usage}
\logpage{[ 1, 2, 0 ]}
\hyperdef{L}{X86A9B6F87E619FFF}{}
{
  

 Specific irreducible representations may be constructed via the methods in
Chapter \ref{Chapter_Irreps}, while lists of irreducible representations with a given degree or level may
be constructed with those in Chapter \ref{Chapter_Lists}. 

 This package uses an \texttt{InfoClass}, \texttt{InfoSL2Reps}. It may be set to \texttt{0} (silent), \texttt{1} (info), or \texttt{2} (verbose). To change it, use 

 \texttt{gap{\textgreater} SetInfoLevel(InfoSL2Reps, k);} 

 }

 }

   
\chapter{\textcolor{Chapter }{Description}}\label{Chapter_Description}
\logpage{[ 2, 0, 0 ]}
\hyperdef{L}{X7BBCB13F82ACC213}{}
{
  

 The group $\mathrm{SL}_2(\mathbb{Z})$ is generated by $\mathfrak{s}$ = \texttt{[[0,1],[-1,0]]} and $\mathfrak{t}$ = \texttt{[[1,1],[0,1]]} (which satisfy the relations $\mathfrak{s}^4 = (\mathfrak{st})^3 = \mathrm{id}$). Thus, any complex representation $\rho$ of $\mathrm{SL}_2(\mathbb{Z})$ on $\mathbb{C}^n$ (where $n \in \mathbb{Z}^+$ is called the \emph{degree} of $\rho$) is defined by the $n \times n$ matrices $S = \rho(\mathfrak{s})$ and $T = \rho(\mathfrak{t})$. 

 This package constructs representations which factor through $\mathrm{SL}_2(\mathbb{Z}/\ell\mathbb{Z})$ for some $\ell \in \mathbb{Z}^+$; the smallest such $\ell$ is called the \emph{level} of the representation. One may equivalently say that the kernel of the
representation is a congruence subgroup. It has been shown that any
representation arising from a modular tensor category has this property \cite{DLN15}. 

 We therefore present representations in the form of a record \texttt{rec(S, T, degree, level, name)} where the name follows the conventions of \cite{NW76}. 

 Note that our definition of $\mathfrak{s}$ follows that of \cite{Nobs1}; other authors prefer the inverse, i.e. $\mathfrak{s}$ = \texttt{[[0,-1],[1,0]]} (under which convention the relations are $\mathfrak{s}^4 = \mathrm{id}$, $(\mathfrak{s}\mathfrak{t})^3 = \mathfrak{s}^2$). When working with that convention, one must invert the $S$ matrices output by this package. 

 Throughout, we denote by $\mathbf{e}$ the map $k \mapsto e^{2 \pi i k}$ (an isomorphism from $\mathbb{Q}/\mathbb{Z}$ to the group of finite roots of unity in $\mathbb{C}$). 

 
\section{\textcolor{Chapter }{Construction}}\label{Chapter_Description_Section_Construction}
\logpage{[ 2, 1, 0 ]}
\hyperdef{L}{X7F6278CD87400D49}{}
{
  

 Any representation $\rho$ of $\mathrm{SL}_2(\mathbb{Z})$ can be decomposed into a direct sum of irreducible representations (irreps).
Further, if $\rho$ has finite level, each irrep can be factorized into a tensor product of irreps
whose levels are powers of distinct primes (using the Chinese remainder
theorem). Therefore, to characterize all representations of $\mathrm{SL}_2(\mathbb{Z})$, it suffices to consider irreps of $\mathrm{SL}_2(\mathbb{Z}/p^\lambda\mathbb{Z})$ for primes $p$ and positive integers $\lambda$. 

 Such representations may be constructed using Weil representations as
described in \cite[Section 1]{Nobs1}. We give a brief summary of the process here. First, if $M$ is any additive abelian group, a \emph{quadratic form} on $M$ is a map $Q : M \to \mathbb{Q}/\mathbb{Z}$ such that $Q(-x) = Q(x)$ for all $x \in M$ and $B(x,y) = Q(x+y) - Q(x) - Q(y)$ defines a $\mathbb{Z}$-bilinear map $M \times M \to \mathbb{Q}/\mathbb{Z}$. 

 Now let $p$ be a prime number and $\lambda \in \mathbb{Z}^+$. Choose a $\mathbb{Z}/p^\lambda\mathbb{Z}$-module $M$ of rank 1 or 2 and a quadratic form $Q$ on $M$ such that the pair $(M,Q)$ is of one of the three types described in Section \ref{Chapter_Description_Section_Weil}. Then the \emph{quadratic module} $(M,Q)$ gives rise to a representation of $\mathrm{SL}_2(\mathbb{Z}/p^\lambda\mathbb{Z})$ on the vector space $V = \mathbb{C}^M$ of complex-valued functions on $M$. This representation is denoted $W(M,Q)$. 

 With a finite number of exceptions, every representation of $\mathrm{SL}_2(\mathbb{Z}/p^\lambda\mathbb{Z})$ may be found as a subrepresentation of $W(M,Q)$ for an appropriate choice of $(M,Q)$ \cite[Hauptsatz 2]{NW76} (the 18 exceptions can be constructed as the tensor product of two such
subrepresentations; these can be generated with \texttt{SL2ExceptionalIrreps} (\ref{SL2ExceptionalIrreps})). 

 The subrepresentations in question are often of the form $W(M,Q,\chi)$, defined as follows. Let $\mathfrak{A}$ be an abelian subgroup of 
\[\operatorname{Aut}(M,Q) = \{ \varepsilon \in \operatorname{Aut}(M) \mid
Q(\varepsilon x) = Q(x) \text{ for all } x \in M\}\]
 and let $\chi \in \widehat{\mathfrak{A}}$ be a 1-dimensional character of $\mathfrak{A}$. Define 
\[V_\chi = \{f \in V \mid f(\varepsilon x) = \chi(\varepsilon) f(x) \text{ for
all } x \in M \text{ and } \varepsilon \in \mathfrak{A}\}~,\]
 which is a $SL_2(\mathbb{Z}/p^\lambda\mathbb{Z})$-invariant subspace of $V$. We then denote by $W(M,Q,\chi)$ the subrepresentation of $W(M,Q)$ on $V_\chi$. 

 In this context, we will frequently refer to a character $\chi$ as being \emph{primitive}, which means that there exists some $\varepsilon \in \mathfrak{A}$ such that $\chi(\varepsilon) \neq 1$ and $\varepsilon$ fixes the submodule $pM$ pointwise (except for one special case of type $R$; for this, see Section \ref{Chapter_Description_Section_Weil_Subsection_Type_R}). This amounts to saying that $\chi$ does not factor through a quadratic module of lower level. 

 All the irreducible representations of $\mathrm{SL}_2(\mathbb{Z})$ of finite level can now be constructed by taking tensor products of these
prime-power irreps. Note that, if two representations are defined by pairs \texttt{[S1, T1]} and \texttt{[S2, T2]}, then their tensor product may be calculated using the GAP command \texttt{KroneckerProduct}, namely as \texttt{[KroneckerProduct(S1,S2), KroneckerProduct(T1,T2)]}. 

 }

 
\section{\textcolor{Chapter }{Weil representation types}}\label{Chapter_Description_Section_Weil}
\logpage{[ 2, 2, 0 ]}
\hyperdef{L}{X861BA4A8800E1A08}{}
{
  

 
\subsection{\textcolor{Chapter }{Type D}}\label{Chapter_Description_Section_Weil_Subsection_Type_D}
\logpage{[ 2, 2, 1 ]}
\hyperdef{L}{X7AEC048F793F1D79}{}
{
  

 Let $p$ be prime and $\lambda \geq 1$. Then the Weil representation arising from the quadratic module with $M = \mathbb{Z}/p^\lambda\mathbb{Z} \oplus \mathbb{Z}/p^\lambda\mathbb{Z}$ and $Q(x,y) = \frac{xy}{p^\lambda}$ is said to be of type $D$ and denoted $D(p,\lambda)$. Information on $(M,Q)$ may be obtained via \texttt{SL2ModuleD} (\ref{SL2ModuleD}), and subrepresentations of $D(p,\lambda)$ with level $p^\lambda$ may be constructed via \texttt{SL2IrrepD} (\ref{SL2IrrepD}). 

 Here we define 
\[\mathfrak{A} = (\mathbb{Z}/p^\lambda\mathbb{Z})^\times\]
 acting on $M$ by multiplication; see \cite[Section 2.1]{NW76}. This group has the following structure. When $p=2$ and $\lambda \geq 3$, it is generated by $\alpha = -1$ and $\zeta = 5$. Otherwise, it is cyclic; in this case, we choose a generator $\alpha$ and (for simplicity) say $\zeta = 1$. 

 A character of $\mathfrak{A}$ is primitive if and only if it is injective on the subgroup $\langle\omicron\rangle \leq \mathfrak{A}$, where $\omicron$ is chosen as follows. If $p=2$ and $\lambda \geq 3$, $\omicron = 5$. If $\lambda = 1$, $\omicron = \alpha$. In all other cases, $\omicron = 1+p$. 

 }

 
\subsection{\textcolor{Chapter }{Type N}}\label{Chapter_Description_Section_Weil_Subsection_Type_N}
\logpage{[ 2, 2, 2 ]}
\hyperdef{L}{X84E15A1E8326AF4F}{}
{
  

 Let $p$ be prime and $\lambda \geq 1$. Then the Weil representation arising from the quadratic module with $M = \mathbb{Z}/p^\lambda\mathbb{Z} \oplus \mathbb{Z}/p^\lambda\mathbb{Z}$ and $Q(x,y) = \frac{x^2 +xy+\frac{1+u}{4}y^2}{p^\lambda}$ (where, for $p \neq 2$, $u$ is chosen so that $u \equiv 3$ mod 4 with $\left(\frac{-u}{p}\right) = -1$, and for $p=2$, $u=3$) is said to be of type $N$ and denoted $N(p,\lambda)$. Information on $(M,Q)$ may be obtained via \texttt{SL2ModuleN} (\ref{SL2ModuleN}), and subrepresentations of $D(p,\lambda)$ with level $p^\lambda$ may be constructed via \texttt{SL2IrrepN} (\ref{SL2IrrepN}). 

 Here we define 
\[\mathfrak{A} = \{\varepsilon \in M^\times \mid \operatorname{Nm}(\varepsilon)
= 1 \}\]
 acting on $M$ by multiplication; see \cite[Section 2.2]{NW76}. This group has the following structure. When $\lambda \geq 2$, it is generated by $\alpha$ and $\zeta$: for $p=2$, $|\alpha| = 2^{\lambda-2}$ and $|\zeta| = 6$, while for $p \neq 2$, $|\alpha| = p^{\lambda-1}$ and $|\zeta| = p+1$. On the other hand, when $\lambda = 1$, $\mathfrak{A}$ is cyclic; in this case, we choose a generator $\zeta$ with $|\zeta| = p+1$ and (for simplicity) say $\alpha = 1$. 

 As a special case, for $p=2$, $\lambda=2$, a character of $\mathfrak{A}$ is primitive if and only if $\chi(-1) = -1$. For all other cases, a character of $\mathfrak{A}$ is primitive if and only if it is injective on the subgroup $\langle\omicron\rangle \leq \mathfrak{A}$, where $\omicron$ is chosen as follows. If $\lambda = 1$, $\omicron = \zeta$; otherwise $\omicron = \alpha$. 

 }

 
\subsection{\textcolor{Chapter }{Type R}}\label{Chapter_Description_Section_Weil_Subsection_Type_R}
\logpage{[ 2, 2, 3 ]}
\hyperdef{L}{X85A14FDA8467E361}{}
{
  

 The structure of $(M,Q)$ of type $R$ depends upon three additional parameters: $\sigma$, $r$, and $t$. The relevant values thereof depend on whether $p=2$, as follows. 

 First, if $p$ is an odd prime, let $\lambda \geq 2$, $\sigma \in \{1, \dots, \lambda\}$, and $r,t \in \{1,u\}$ with $u$ a quadratic non-residue mod $p$. Then define $M = \mathbb{Z}/p^\lambda\mathbb{Z} \oplus
\mathbb{Z}/p^{\lambda-\sigma}\mathbb{Z}$ and $Q(x,y) = \frac{r(x^2 + p^\sigma t y^2)}{p^\lambda}$. 

 On the other hand, if $p=2$, let $\lambda \geq 2$, $\sigma \in \{0, \dots, \lambda-2\}$ and $r,t \in \{1,3,5,7\}$. Then define $M = \mathbb{Z}/2^{\lambda-1}\mathbb{Z} \oplus
\mathbb{Z}/2^{\lambda-\sigma-1}\mathbb{Z}$ and $Q(x,y) = \frac{r(x^2 + 2^\sigma t y^2)}{2^\lambda}$. 

 In either case, the resulting representation is said to be of type $R$ and denoted $R(p,\lambda,\sigma,r,t)$. Information on $(M,Q)$ may be obtained via \texttt{SL2ModuleR} (\ref{SL2ModuleR}), and subrepresentations of $R(p,\lambda,\sigma,r,t)$ with level $p^\lambda$ may be constructed via \texttt{SL2IrrepR} (\ref{SL2IrrepR}). Note that if $\sigma = \lambda$ for $p \neq 2$, then the second factor of $M$ is trivial (and hence $t$ is irrelevant); this special case is handled by \texttt{SL2IrrepRUnary} (\ref{SL2IrrepRUnary}). }

 }

 }

   
\chapter{\textcolor{Chapter }{Lists of representations}}\label{Chapter_Lists}
\logpage{[ 3, 0, 0 ]}
\hyperdef{L}{X8168FF017B2C0BB2}{}
{
  
\section{\textcolor{Chapter }{Lists by degree}}\label{Chapter_Lists_Section_Degree}
\logpage{[ 3, 1, 0 ]}
\hyperdef{L}{X80406B597E11D7C6}{}
{
  

\subsection{\textcolor{Chapter }{SL2PrimePowerIrrepsOfDegree}}
\logpage{[ 3, 1, 1 ]}\nobreak
\hyperdef{L}{X7AFFB022821949A7}{}
{\noindent\textcolor{FuncColor}{$\triangleright$\enspace\texttt{SL2PrimePowerIrrepsOfDegree({\mdseries\slshape degree})\index{SL2PrimePowerIrrepsOfDegree@\texttt{SL2PrimePowerIrrepsOfDegree}}
\label{SL2PrimePowerIrrepsOfDegree}
}\hfill{\scriptsize (function)}}\\
\textbf{\indent Returns:\ }
a list of records of the form \texttt{rec(S, T, degree, level, name)} 



 Constructs a list of all irreps of $\mathrm{SL}_2(\mathbb{Z})$ that are exactly the given degree and have prime power level. }

 

\subsection{\textcolor{Chapter }{SL2PrimePowerIrrepsOfDegreeAtMost}}
\logpage{[ 3, 1, 2 ]}\nobreak
\hyperdef{L}{X7CF672A47E035702}{}
{\noindent\textcolor{FuncColor}{$\triangleright$\enspace\texttt{SL2PrimePowerIrrepsOfDegreeAtMost({\mdseries\slshape max{\textunderscore}degree})\index{SL2PrimePowerIrrepsOfDegreeAtMost@\texttt{SL2PrimePowerIrrepsOfDegreeAtMost}}
\label{SL2PrimePowerIrrepsOfDegreeAtMost}
}\hfill{\scriptsize (function)}}\\
\textbf{\indent Returns:\ }
a list of records of the form \texttt{rec(S, T, degree, level, name)} 



 Constructs a list of all irreps of $\mathrm{SL}_2(\mathbb{Z})$ that are at most the given degree and have prime power level. }

 

\subsection{\textcolor{Chapter }{SL2IrrepsOfDegree}}
\logpage{[ 3, 1, 3 ]}\nobreak
\hyperdef{L}{X7CC7D26D7E37A1B9}{}
{\noindent\textcolor{FuncColor}{$\triangleright$\enspace\texttt{SL2IrrepsOfDegree({\mdseries\slshape degree})\index{SL2IrrepsOfDegree@\texttt{SL2IrrepsOfDegree}}
\label{SL2IrrepsOfDegree}
}\hfill{\scriptsize (function)}}\\
\textbf{\indent Returns:\ }
a list of records of the form \texttt{rec(S, T, degree, level, name)} 



 Constructs a list of all irreps of $\mathrm{SL}_2(\mathbb{Z})$ that are exactly the given degree. }

 

\subsection{\textcolor{Chapter }{SL2IrrepsOfDegreeAtMost}}
\logpage{[ 3, 1, 4 ]}\nobreak
\hyperdef{L}{X78044A4683369ADC}{}
{\noindent\textcolor{FuncColor}{$\triangleright$\enspace\texttt{SL2IrrepsOfDegreeAtMost({\mdseries\slshape degree})\index{SL2IrrepsOfDegreeAtMost@\texttt{SL2IrrepsOfDegreeAtMost}}
\label{SL2IrrepsOfDegreeAtMost}
}\hfill{\scriptsize (function)}}\\
\textbf{\indent Returns:\ }
a list of records of the form \texttt{rec(S, T, degree, level, name)} 



 Constructs a list of all irreps of $\mathrm{SL}_2(\mathbb{Z})$ that are at most the given degree. }

 }

 
\section{\textcolor{Chapter }{Lists by level}}\label{Chapter_Lists_Section_Level}
\logpage{[ 3, 2, 0 ]}
\hyperdef{L}{X78C583957D3FF6ED}{}
{
  

\subsection{\textcolor{Chapter }{SL2PrimePowerIrrepsOfLevel}}
\logpage{[ 3, 2, 1 ]}\nobreak
\hyperdef{L}{X7A0D2DB780D99B11}{}
{\noindent\textcolor{FuncColor}{$\triangleright$\enspace\texttt{SL2PrimePowerIrrepsOfLevel({\mdseries\slshape p, lambda})\index{SL2PrimePowerIrrepsOfLevel@\texttt{SL2PrimePowerIrrepsOfLevel}}
\label{SL2PrimePowerIrrepsOfLevel}
}\hfill{\scriptsize (function)}}\\
\textbf{\indent Returns:\ }
a list of records of the form \texttt{rec(S, T, degree, level, name)} 



 Constructs a list of all irreps of $\mathrm{SL}_2(\mathbb{Z})$ with level exactly $p^\lambda$. }

 }

 
\section{\textcolor{Chapter }{Lists of exceptional representations}}\label{Chapter_Lists_Section_Exceptions}
\logpage{[ 3, 3, 0 ]}
\hyperdef{L}{X80400C6D79D4D0D6}{}
{
  

\subsection{\textcolor{Chapter }{SL2ExceptionalIrreps}}
\logpage{[ 3, 3, 1 ]}\nobreak
\hyperdef{L}{X85078197803C9848}{}
{\noindent\textcolor{FuncColor}{$\triangleright$\enspace\texttt{SL2ExceptionalIrreps({\mdseries\slshape arg})\index{SL2ExceptionalIrreps@\texttt{SL2ExceptionalIrreps}}
\label{SL2ExceptionalIrreps}
}\hfill{\scriptsize (function)}}\\
\textbf{\indent Returns:\ }
a list of records of the form \texttt{rec(S, T, degree, level, name)} 



 Constructs a list of the 18 exceptional irreps of $\mathrm{SL}_2(\mathbb{Z})$. }

 }

 }

   
\chapter{\textcolor{Chapter }{Methods for testing}}\label{Chapter_Testing}
\logpage{[ 4, 0, 0 ]}
\hyperdef{L}{X794C1A137F8FA14D}{}
{
  
\section{\textcolor{Chapter }{Testing}}\label{Chapter_Testing_Section_Testing}
\logpage{[ 4, 1, 0 ]}
\hyperdef{L}{X7DE7E7187BE24368}{}
{
  

\subsection{\textcolor{Chapter }{SL2WithConjClasses}}
\logpage{[ 4, 1, 1 ]}\nobreak
\hyperdef{L}{X801680187A8CA462}{}
{\noindent\textcolor{FuncColor}{$\triangleright$\enspace\texttt{SL2WithConjClasses({\mdseries\slshape p, ld})\index{SL2WithConjClasses@\texttt{SL2WithConjClasses}}
\label{SL2WithConjClasses}
}\hfill{\scriptsize (function)}}\\
\textbf{\indent Returns:\ }
the group $\mathrm{SL}_2(\mathbb{Z}/p^\lambda\mathbb{Z})$ with conjugacy classes set to the format we use. 



 

 }

 

\subsection{\textcolor{Chapter }{SL2ChiST}}
\logpage{[ 4, 1, 2 ]}\nobreak
\hyperdef{L}{X7A97C7C77C7608D0}{}
{\noindent\textcolor{FuncColor}{$\triangleright$\enspace\texttt{SL2ChiST({\mdseries\slshape S, T, p, ld})\index{SL2ChiST@\texttt{SL2ChiST}}
\label{SL2ChiST}
}\hfill{\scriptsize (function)}}\\
\textbf{\indent Returns:\ }
a list representing a character of $\mathrm{SL}_2(\mathbb{Z}/p^\lambda\mathbb{Z})$ 



 Converts the modular data $(S,T)$, which must have level dividing $p^\lambda$, into a character of $\mathrm{SL}_2(\mathbb{Z}/p^\lambda\mathbb{Z})$, presented in a form matching the conjugacy classes used in \texttt{SL2WithConjClasses}. }

 

\subsection{\textcolor{Chapter }{SL2IrrepPositionTest}}
\logpage{[ 4, 1, 3 ]}\nobreak
\hyperdef{L}{X7FB0262F7E0F1B95}{}
{\noindent\textcolor{FuncColor}{$\triangleright$\enspace\texttt{SL2IrrepPositionTest({\mdseries\slshape p, lambda})\index{SL2IrrepPositionTest@\texttt{SL2IrrepPositionTest}}
\label{SL2IrrepPositionTest}
}\hfill{\scriptsize (function)}}\\
\textbf{\indent Returns:\ }
a boolean 



 Constructs and tests all irreps of level dividing $p^\lambda$ by checking their positions in \texttt{Irr(G)}. }

 }

 }

   
\chapter{\textcolor{Chapter }{Irreducible representations of prime-power level}}\label{Chapter_Irreps}
\logpage{[ 5, 0, 0 ]}
\hyperdef{L}{X7C4165447B8AB223}{}
{
  Methods for generating individual irreducible representations of $\mathrm{SL}_2(\mathbb{Z}/p^\lambda\mathbb{Z})$ for a given level $p^\lambda$. 

 In each case (except the unary type $R$, for which see \texttt{SL2IrrepRUnary} (\ref{SL2IrrepRUnary})), the underlying module $M$ is of rank $2$, so its elements have the form $(x,y)$ and are thus represented by lists $[x,y]$. 
\section{\textcolor{Chapter }{Representations of type D}}\label{Chapter_Irreps_Section_Representations_of_type_D}
\logpage{[ 5, 1, 0 ]}
\hyperdef{L}{X82DA126D7F755B1C}{}
{
  

 See Section \ref{Chapter_Description_Section_Weil_Subsection_Type_D}. 

\subsection{\textcolor{Chapter }{SL2ModuleD}}
\logpage{[ 5, 1, 1 ]}\nobreak
\hyperdef{L}{X845D92CB7841CB0B}{}
{\noindent\textcolor{FuncColor}{$\triangleright$\enspace\texttt{SL2ModuleD({\mdseries\slshape p, ld})\index{SL2ModuleD@\texttt{SL2ModuleD}}
\label{SL2ModuleD}
}\hfill{\scriptsize (function)}}\\
\textbf{\indent Returns:\ }
a record \texttt{rec(Agrp, Bp, Char, IsPrim)} describing $(M,Q)$ 



 Constructs information about the underlying quadratic module $(M,Q)$ of type $D$, for $p$ a prime and $\lambda \geq 1$. 

 \texttt{Agrp} is a list describing the elements of $\mathfrak{A}$. Each element $a \in \mathfrak{A}$ is represented in \texttt{Agrp} by a list \texttt{[v, a, a{\textunderscore}inv]}, where \texttt{v} is a list defined by $a = \alpha^{\mathtt{v[1]}} \zeta^{\mathtt{v[2]}}$. Note that $\zeta$ is trivial, and hence \texttt{v[2]} is irrelevant, when $\mathfrak{A}$ is cyclic. 

 \texttt{Bp} is a list of representatives for the $\mathfrak{A}$-orbits on $M^\times$, which correspond to a basis for the $\mathrm{SL}_2(\mathbb{Z}/p^\lambda\mathbb{Z})$-invariant subspace associated to any primitive character $\chi \in \widehat{\mathfrak{A}}$ with $\chi^2 \not\equiv 1$. For other characters, we must use different bases which are particular to
each case. 

 \texttt{Char(i,j)} converts two integers $i$, $j$ to a function representing a character of $\mathfrak{A}$. Each character in $\hat{\mathfrak{A}}$ is of the form $\chi_{i,j}$, given by
\texttt{\symbol{92}}[\texttt{\symbol{92}}chi{\textunderscore}\texttt{\symbol{123}}i,j\texttt{\symbol{125}}(\texttt{\symbol{92}}alpha\texttt{\symbol{94}}\texttt{\symbol{123}}v\texttt{\symbol{125}}\texttt{\symbol{92}}zeta\texttt{\symbol{94}}\texttt{\symbol{123}}w\texttt{\symbol{125}})
\texttt{\symbol{92}}mapsto
\texttt{\symbol{92}}mathbf\texttt{\symbol{123}}e\texttt{\symbol{125}}\texttt{\symbol{92}}left(\texttt{\symbol{92}}frac\texttt{\symbol{123}}vi\texttt{\symbol{125}}\texttt{\symbol{123}}|\texttt{\symbol{92}}alpha|\texttt{\symbol{125}}\texttt{\symbol{92}}right)
\texttt{\symbol{92}}mathbf\texttt{\symbol{123}}e\texttt{\symbol{125}}\texttt{\symbol{92}}left(\texttt{\symbol{92}}frac\texttt{\symbol{123}}wj\texttt{\symbol{125}}\texttt{\symbol{123}}|\texttt{\symbol{92}}zeta|\texttt{\symbol{125}}\texttt{\symbol{92}}right)\texttt{\symbol{126}}.\texttt{\symbol{92}}]
Note that $j$ is irrelevant when $\mathfrak{A}$ is cyclic. 

 \texttt{IsPrim(chi)} tests whether the output of \texttt{Char(i,j)} represents a primitive character. }

 

\subsection{\textcolor{Chapter }{SL2IrrepD}}
\logpage{[ 5, 1, 2 ]}\nobreak
\hyperdef{L}{X7FDB517981A2C091}{}
{\noindent\textcolor{FuncColor}{$\triangleright$\enspace\texttt{SL2IrrepD({\mdseries\slshape p, ld, chi{\textunderscore}index})\index{SL2IrrepD@\texttt{SL2IrrepD}}
\label{SL2IrrepD}
}\hfill{\scriptsize (function)}}\\
\textbf{\indent Returns:\ }
a list of lists of the form $[S,T]$ 



 Constructs the modular data for the irreducible representation(s) of type $D$ with level $p^\lambda$, for $p$ a prime and $\lambda \geq 1$, corresponding to the character $\chi$ indexed by \texttt{chi{\textunderscore}index = [i,j]} (see the discussion of \texttt{Char(i,j)} in \texttt{SL2ModuleD} (\ref{SL2ModuleD})). 

 Depending on the parameters, $W(M,Q)$ will contain either 1 or 2 such irreps. }

 }

 
\section{\textcolor{Chapter }{Representations of type N}}\label{Chapter_Irreps_Section_Representations_of_type_N}
\logpage{[ 5, 2, 0 ]}
\hyperdef{L}{X7CD74CFC84B38042}{}
{
  

 See Section \ref{Chapter_Description_Section_Weil_Subsection_Type_N}. 

\subsection{\textcolor{Chapter }{SL2ModuleN}}
\logpage{[ 5, 2, 1 ]}\nobreak
\hyperdef{L}{X7A50CC5A7933E207}{}
{\noindent\textcolor{FuncColor}{$\triangleright$\enspace\texttt{SL2ModuleN({\mdseries\slshape p, ld})\index{SL2ModuleN@\texttt{SL2ModuleN}}
\label{SL2ModuleN}
}\hfill{\scriptsize (function)}}\\
\textbf{\indent Returns:\ }
a record \texttt{rec(Agrp, Bp, Char, IsPrim, Nm, Prod)} describing $(M,Q)$ 



 Constructs information about the underlying quadratic module $(M,Q)$ of type $N$, for $p$ a prime and $\lambda \geq 1$. 

 \texttt{Agrp} is a list describing the elements of $\mathfrak{A}$. Each element $a \in \mathfrak{A}$ is represented in \texttt{Agrp} by a list \texttt{[v, a]}, where \texttt{v} is a list defined by $a = \alpha^{\mathtt{v[1]}} \zeta^{\mathtt{v[2]}}$. Note that $\alpha$ is trivial, and hence \texttt{v[1]} is irrelevant, when $\mathfrak{A}$ is cyclic. 

 \texttt{Bp} is a list of representatives for the $\mathfrak{A}$-orbits on $M^\times$, which correspond to a basis for the $\mathrm{SL}_2(\mathbb{Z}/p^\lambda\mathbb{Z})$-invariant subspace associated to any primitive character $\chi \in \widehat{\mathfrak{A}}$ with $\chi^2 \not\equiv 1$. For other characters, we must use different bases which are particular to
each case. 

 \texttt{Char(i,j)} converts two integers $i$, $j$ to a function representing a character of $\mathfrak{A}$. Each character in $\hat{\mathfrak{A}}$ is of the form $\chi_{i,j}$, given by
\texttt{\symbol{92}}[\texttt{\symbol{92}}chi{\textunderscore}\texttt{\symbol{123}}i,j\texttt{\symbol{125}}(\texttt{\symbol{92}}alpha\texttt{\symbol{94}}\texttt{\symbol{123}}v\texttt{\symbol{125}}\texttt{\symbol{92}}zeta\texttt{\symbol{94}}\texttt{\symbol{123}}w\texttt{\symbol{125}})
\texttt{\symbol{92}}mapsto
\texttt{\symbol{92}}mathbf\texttt{\symbol{123}}e\texttt{\symbol{125}}\texttt{\symbol{92}}left(\texttt{\symbol{92}}frac\texttt{\symbol{123}}vi\texttt{\symbol{125}}\texttt{\symbol{123}}|\texttt{\symbol{92}}alpha|\texttt{\symbol{125}}\texttt{\symbol{92}}right)
\texttt{\symbol{92}}mathbf\texttt{\symbol{123}}e\texttt{\symbol{125}}\texttt{\symbol{92}}left(\texttt{\symbol{92}}frac\texttt{\symbol{123}}wj\texttt{\symbol{125}}\texttt{\symbol{123}}|\texttt{\symbol{92}}zeta|\texttt{\symbol{125}}\texttt{\symbol{92}}right)\texttt{\symbol{126}}.\texttt{\symbol{92}}]
Note that $i$ is irrelevant in the cases where $\mathfrak{A}$ is cyclic. 

 \texttt{IsPrim(chi)} tests whether the output of \texttt{Char(i,j)} represents a primitive character. 

 \texttt{Nm(a)} and \texttt{Prod(a,b)} are the norm and product functions on $M$, respectively. }

 

\subsection{\textcolor{Chapter }{SL2IrrepN}}
\logpage{[ 5, 2, 2 ]}\nobreak
\hyperdef{L}{X81D60FE878F02838}{}
{\noindent\textcolor{FuncColor}{$\triangleright$\enspace\texttt{SL2IrrepN({\mdseries\slshape p, ld, chi{\textunderscore}index})\index{SL2IrrepN@\texttt{SL2IrrepN}}
\label{SL2IrrepN}
}\hfill{\scriptsize (function)}}\\
\textbf{\indent Returns:\ }
a list of lists of the form $[S,T]$ 



 Constructs the modular data for the irreducible representation(s) of type $N$ with level $p^\lambda$, for $p$ a prime and $\lambda \geq 1$, corresponding to the character $\chi$ indexed by \texttt{chi{\textunderscore}index = [i,j]} (see the discussion of \texttt{Char(i,j)} in \texttt{SL2ModuleN} (\ref{SL2ModuleN})). 

 Depending on the parameters, $W(M,Q)$ will contain either 1 or 2 such irreps. }

 }

 
\section{\textcolor{Chapter }{Representations of type R}}\label{Chapter_Irreps_Section_Representations_of_type_R}
\logpage{[ 5, 3, 0 ]}
\hyperdef{L}{X7D9759387C499FA0}{}
{
  

 See Section \ref{Chapter_Description_Section_Weil_Subsection_Type_R}. 

\subsection{\textcolor{Chapter }{SL2ModuleR}}
\logpage{[ 5, 3, 1 ]}\nobreak
\hyperdef{L}{X7B10D99E7AEAC411}{}
{\noindent\textcolor{FuncColor}{$\triangleright$\enspace\texttt{SL2ModuleR({\mdseries\slshape p, ld, sigma, r, t})\index{SL2ModuleR@\texttt{SL2ModuleR}}
\label{SL2ModuleR}
}\hfill{\scriptsize (function)}}\\
\textbf{\indent Returns:\ }
a record \texttt{rec(Agrp, Bp, Char, IsPrim, Nm, Ord, Prod, c, tM)} describing $(M,Q)$ a record \texttt{rec(Agrp, Char, IsPrim, Nm, Ord, Prod, c, tM)} describing $(M,Q)$ 



 Constructs information about the underlying quadratic module $(M,Q)$ of type $R$, for $p$ a prime. The additional parameters $\lambda$, $\sigma$, $r$, and $t$ should be integers chosen as follows. 

 If $p$ is an odd prime, let $\lambda \geq 2$, $\sigma \in \{1, \dots, \lambda - 1\}$, and $r,t \in \{1,u\}$ with $u$ a quadratic non-residue mod $p$. Note that $\sigma = \lambda$ is a valid choice for type $R$, however, this gives the unary case, and so is not handled by this function,
as it is decomposed in a different way; for this case, use \texttt{SL2IrrepRUnary} (\ref{SL2IrrepRUnary}) instead. 

 If $p=2$, let $\lambda \geq 2$, $\sigma \in \{0, \dots, \lambda-2\}$ and $r,t \in \{1,3,5,7\}$. 

 \texttt{Agrp} is a list describing the elements of
\texttt{\symbol{92}}[\texttt{\symbol{92}}mathfrak\texttt{\symbol{123}}A\texttt{\symbol{125}}
= \texttt{\symbol{92}}\texttt{\symbol{123}}\texttt{\symbol{92}}varepsilon
\texttt{\symbol{92}}in M\texttt{\symbol{94}}\texttt{\symbol{92}}times
\texttt{\symbol{92}}mid
\texttt{\symbol{92}}operatorname\texttt{\symbol{123}}Nm\texttt{\symbol{125}}(\texttt{\symbol{92}}varepsilon)
= 1 \texttt{\symbol{92}}\texttt{\symbol{125}}\texttt{\symbol{92}}] (see \cite[Section 2.3 - 2.5]{NW76}). The group $\mathfrak{A}$ has the following form. 

 First consider the special case $p = 2$ and $\sigma = \lambda - 2$. If $\lambda = 4$, then $\mathfrak{A} \cong \mathbb{Z}/2\mathbb{Z}$, generated by $\zeta = 7$; for simplicity we say $\alpha = 1$. Otherwise, $\mathfrak{A} = \langle\alpha\rangle \times \langle\zeta\rangle$ with $\zeta = -1$ and $\alpha \neq \zeta$ of order 2. 

 Each element $a$ of $\mathfrak{A}$ is represented in \texttt{Agrp} by a list \texttt{[v, a]}, where \texttt{v} is a list defined by $a = \alpha^{\mathtt{v[1]}} \zeta^{\mathtt{v[2]}}$. 

 \texttt{Bp} is a list of representatives for the $\mathfrak{A}$-orbits on $M^\times$, which correspond to a basis for the $\mathrm{SL}_2(\mathbb{Z}/p^\lambda\mathbb{Z})$-invariant subspace associated to any primitive character $\chi \in \widehat{\mathfrak{A}}$ with $\chi^2 \not\equiv 1$. For other characters, we must use different bases which are particular to
each case. 

 \texttt{Char(i,j)} converts two integers $i$, $j$ to a function representing a character of $\mathfrak{A}$. Each character in $\hat{\mathfrak{A}}$ is of the form $\chi_{i,j}$, given by
\texttt{\symbol{92}}[\texttt{\symbol{92}}chi{\textunderscore}\texttt{\symbol{123}}i,j\texttt{\symbol{125}}(\texttt{\symbol{92}}alpha\texttt{\symbol{94}}\texttt{\symbol{123}}v\texttt{\symbol{125}}\texttt{\symbol{92}}zeta\texttt{\symbol{94}}\texttt{\symbol{123}}w\texttt{\symbol{125}})
\texttt{\symbol{92}}mapsto
\texttt{\symbol{92}}mathbf\texttt{\symbol{123}}e\texttt{\symbol{125}}\texttt{\symbol{92}}left(\texttt{\symbol{92}}frac\texttt{\symbol{123}}vi\texttt{\symbol{125}}\texttt{\symbol{123}}|\texttt{\symbol{92}}alpha|\texttt{\symbol{125}}\texttt{\symbol{92}}right)
\texttt{\symbol{92}}mathbf\texttt{\symbol{123}}e\texttt{\symbol{125}}\texttt{\symbol{92}}left(\texttt{\symbol{92}}frac\texttt{\symbol{123}}wj\texttt{\symbol{125}}\texttt{\symbol{123}}|\texttt{\symbol{92}}zeta|\texttt{\symbol{125}}\texttt{\symbol{92}}right)\texttt{\symbol{126}}.\texttt{\symbol{92}}]
Note that $i$ is irrelevant in the cases where $\mathfrak{A}$ is cyclic. 

 \texttt{IsPrim(chi)} tests whether the output of \texttt{Char(i,j)} represents a primitive character. As a special case, for $p=2$, $\lambda \geq 5$, $\sigma = \lambda - 2$, a character is primitive if $\chi(\alpha) = -1$ (this case differs from the usual definition of primitivity; see \cite[Section 2.5]{NW76}). For all other cases, a character is primitive if it is injective on $\langle\omicron\rangle \leq \mathfrak{A}$, where $\omicron$ is defined as follows. 

 First suppose $p = 2$. If $\lambda = 4$, $\sigma = 2$, then $\omicron = \alpha$. 

 \texttt{Nm(a)}, \texttt{Ord(a)}, and \texttt{Prod(a,b)} are the norm, order, and product functions on $M$, respectively. 

 \texttt{c} is a scalar used in calculating the $S$-matrix; namely $c = \frac{1}{|M|} \sum_{x \in M} \mathbf{e}(Q(x))$. Note that this is equal to $S_Q(-1) / \sqrt{|M|}$, where $S_Q(-1) = \frac{1}{\sqrt{|M|}} \sum_{x \in M} \mathbf{e}(Q(x))$ is also known as the central charge. 

 \texttt{tM} is a list describing the elements of the group $M - pM$. Constructs information about the underlying quadratic module $(M,Q)$ of type $R$, for $p$ a prime, $\lambda \geq 1$. See 

 \texttt{Agrp} describes the elements of $\mathfrak{A} = \{\varepsilon \in M^\times \mid \operatorname{Nm}(\varepsilon)
= 1 \}$ (see \cite[Section 2.3 - 2.5]{NW76}). 

 Representatives for the $\mathfrak{A}$-orbits on $M^\times$ can depend on the choice of character, even for primitive characters $\chi$ with $\chi^2 \not\equiv 1$. Thus, we cannot provide them here, and they are instead calculated by \texttt{SL2IrrepR} (\ref{SL2IrrepR}). 

 \texttt{Char(i,j)} converts the \texttt{chi{\textunderscore}index} used in \texttt{SL2IrrepR} (\ref{SL2IrrepR}) to a function. 

 \texttt{IsPrim(chi)} tests whether a given character (e.g. from \texttt{Char}) is primitive. 

 \texttt{Nm(a)}, \texttt{Ord(a)}, and \texttt{Prod(a,b)} are the norm, order, and product functions on $M$, respectively. 

 \texttt{c} is a scalar used in calculating the $S$-matrix; namely $c = \frac{1}{|M|} \sum_{x \in M} \mathbf{e}(Q(x))$. 

 \texttt{tM} is the group $M - pM$. }

 

\subsection{\textcolor{Chapter }{SL2IrrepR}}
\logpage{[ 5, 3, 2 ]}\nobreak
\hyperdef{L}{X80961A2C7C5F632E}{}
{\noindent\textcolor{FuncColor}{$\triangleright$\enspace\texttt{SL2IrrepR({\mdseries\slshape p, ld, sigma, r, t, chi{\textunderscore}index})\index{SL2IrrepR@\texttt{SL2IrrepR}}
\label{SL2IrrepR}
}\hfill{\scriptsize (function)}}\\
\textbf{\indent Returns:\ }
a list of lists of the form $[S,T]$ 



 Constructs the modular data for the irreducible representation(s) of type $R$ with parameters $p$, $\lambda$, $\sigma$, $r$, $t$ as described in \texttt{SL2ModuleN} (\ref{SL2ModuleN}), corresponding to the character $\chi$ indexed by \texttt{chi{\textunderscore}index = [i,j]} (see the discussions of $\sigma$, $r$, $t$, and \texttt{Char(i,j)} in \texttt{SL2ModuleN} (\ref{SL2ModuleN})). 

 Depending on the parameters, $W(M,Q)$ will contain either 1 or 2 such irreps. 

 If $\sigma = \lambda$ for $p \neq 2$, then the second factor of $M$ is trivial (and hence $t$ is irrelevant) so this falls through to \texttt{SL2IrrepRUnary} (\ref{SL2IrrepRUnary}). }

 

\subsection{\textcolor{Chapter }{SL2IrrepRUnary}}
\logpage{[ 5, 3, 3 ]}\nobreak
\hyperdef{L}{X7C94E3007A1BEE85}{}
{\noindent\textcolor{FuncColor}{$\triangleright$\enspace\texttt{SL2IrrepRUnary({\mdseries\slshape p, ld, r})\index{SL2IrrepRUnary@\texttt{SL2IrrepRUnary}}
\label{SL2IrrepRUnary}
}\hfill{\scriptsize (function)}}\\
\textbf{\indent Returns:\ }
a list of lists of the form $[S,T]$ 



 Constructs the modular data for the irreducible representation(s) of unary
type $R$ (that is, the special case where $\sigma = \lambda$) with $p$ an odd prime, $\lambda$ a positive integer, and $r \in \{1,u\}$ with $u$ a quadratic non-residue mod $p$. 

 In this case, $W(M,Q)$ always contains exactly 2 such irreps. }

 }

 }

 \def\bibname{References\logpage{[ "Bib", 0, 0 ]}
\hyperdef{L}{X7A6F98FD85F02BFE}{}
}

\bibliographystyle{alpha}
\bibliography{SL2Reps.bib}

\addcontentsline{toc}{chapter}{References}

\def\indexname{Index\logpage{[ "Ind", 0, 0 ]}
\hyperdef{L}{X83A0356F839C696F}{}
}

\cleardoublepage
\phantomsection
\addcontentsline{toc}{chapter}{Index}


\printindex

\immediate\write\pagenrlog{["Ind", 0, 0], \arabic{page},}
\immediate\write\pagenrlog{["Ind", 0, 0], \arabic{page},}
\immediate\write\pagenrlog{["Ind", 0, 0], \arabic{page},}
\immediate\write\pagenrlog{["Ind", 0, 0], \arabic{page},}
\immediate\write\pagenrlog{["Ind", 0, 0], \arabic{page},}
\immediate\write\pagenrlog{["Ind", 0, 0], \arabic{page},}
\newpage
\immediate\write\pagenrlog{["End"], \arabic{page}];}
\immediate\closeout\pagenrlog
\end{document}
