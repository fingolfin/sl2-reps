% generated by GAPDoc2LaTeX from XML source (Frank Luebeck)
\documentclass[a4paper,11pt]{report}

\usepackage[top=37mm,bottom=37mm,left=27mm,right=27mm]{geometry}
\sloppy
\pagestyle{myheadings}
\usepackage{amssymb,amsmath}
\usepackage[utf8]{inputenc}
\usepackage{makeidx}
\makeindex
\usepackage{color}
\definecolor{FireBrick}{rgb}{0.5812,0.0074,0.0083}
\definecolor{RoyalBlue}{rgb}{0.0236,0.0894,0.6179}
\definecolor{RoyalGreen}{rgb}{0.0236,0.6179,0.0894}
\definecolor{RoyalRed}{rgb}{0.6179,0.0236,0.0894}
\definecolor{LightBlue}{rgb}{0.8544,0.9511,1.0000}
\definecolor{Black}{rgb}{0.0,0.0,0.0}

\definecolor{linkColor}{rgb}{0.0,0.0,0.554}
\definecolor{citeColor}{rgb}{0.0,0.0,0.554}
\definecolor{fileColor}{rgb}{0.0,0.0,0.554}
\definecolor{urlColor}{rgb}{0.0,0.0,0.554}
\definecolor{promptColor}{rgb}{0.0,0.0,0.589}
\definecolor{brkpromptColor}{rgb}{0.589,0.0,0.0}
\definecolor{gapinputColor}{rgb}{0.589,0.0,0.0}
\definecolor{gapoutputColor}{rgb}{0.0,0.0,0.0}

%%  for a long time these were red and blue by default,
%%  now black, but keep variables to overwrite
\definecolor{FuncColor}{rgb}{0.0,0.0,0.0}
%% strange name because of pdflatex bug:
\definecolor{Chapter }{rgb}{0.0,0.0,0.0}
\definecolor{DarkOlive}{rgb}{0.1047,0.2412,0.0064}


\usepackage{fancyvrb}

\usepackage{mathptmx,helvet}
\usepackage[T1]{fontenc}
\usepackage{textcomp}


\usepackage[
            pdftex=true,
            bookmarks=true,        
            a4paper=true,
            pdftitle={Written with GAPDoc},
            pdfcreator={LaTeX with hyperref package / GAPDoc},
            colorlinks=true,
            backref=page,
            breaklinks=true,
            linkcolor=linkColor,
            citecolor=citeColor,
            filecolor=fileColor,
            urlcolor=urlColor,
            pdfpagemode={UseNone}, 
           ]{hyperref}

\newcommand{\maintitlesize}{\fontsize{50}{55}\selectfont}

% write page numbers to a .pnr log file for online help
\newwrite\pagenrlog
\immediate\openout\pagenrlog =\jobname.pnr
\immediate\write\pagenrlog{PAGENRS := [}
\newcommand{\logpage}[1]{\protect\write\pagenrlog{#1, \thepage,}}
%% were never documented, give conflicts with some additional packages

\newcommand{\GAP}{\textsf{GAP}}

%% nicer description environments, allows long labels
\usepackage{enumitem}
\setdescription{style=nextline}

%% depth of toc
\setcounter{tocdepth}{1}





%% command for ColorPrompt style examples
\newcommand{\gapprompt}[1]{\color{promptColor}{\bfseries #1}}
\newcommand{\gapbrkprompt}[1]{\color{brkpromptColor}{\bfseries #1}}
\newcommand{\gapinput}[1]{\color{gapinputColor}{#1}}


\begin{document}

\logpage{[ 0, 0, 0 ]}
\begin{titlepage}
\mbox{}\vfill

\begin{center}{\maintitlesize \textbf{ SL2Reps \mbox{}}}\\
\vfill

\hypersetup{pdftitle= SL2Reps }
\markright{\scriptsize \mbox{}\hfill  SL2Reps  \hfill\mbox{}}
{\Huge \textbf{ Constructs representations of SL2(Z). \mbox{}}}\\
\vfill

{\Huge  0.1 \mbox{}}\\[1cm]
{ 24 September 2021 \mbox{}}\\[1cm]
\mbox{}\\[2cm]
{\Large \textbf{ Siu-Hung Ng\\
  \mbox{}}}\\
{\Large \textbf{ Yilong Wang\\
  \mbox{}}}\\
{\Large \textbf{ Samuel Wilson\\
  \mbox{}}}\\
\hypersetup{pdfauthor= Siu-Hung Ng\\
  ;  Yilong Wang\\
  ;  Samuel Wilson\\
  }
\end{center}\vfill

\mbox{}\\
{\mbox{}\\
\small \noindent \textbf{ Siu-Hung Ng\\
  }  Email: \href{mailto://rng@math.lsu.edu} {\texttt{rng@math.lsu.edu}}}\\
{\mbox{}\\
\small \noindent \textbf{ Yilong Wang\\
  }  Email: \href{mailto://wyl@bimsa.cn} {\texttt{wyl@bimsa.cn}}}\\
{\mbox{}\\
\small \noindent \textbf{ Samuel Wilson\\
  }  Email: \href{mailto://swil311@lsu.edu} {\texttt{swil311@lsu.edu}}}\\
\end{titlepage}

\newpage\setcounter{page}{2}
\newpage

\def\contentsname{Contents\logpage{[ 0, 0, 1 ]}}

\tableofcontents
\newpage

     
\chapter{\textcolor{Chapter }{Introduction}}\label{Chapter_Introduction}
\logpage{[ 1, 0, 0 ]}
\hyperdef{L}{X7DFB63A97E67C0A1}{}
{
  

 This package, \texttt{SL2Reps}, provides methods for constructing and testing matrix presentations of the
representations of $\mathrm{SL}_2(\mathbb{Z})$. 

 Irreducible representations of prime-power level are constructed individually
by means of Weyl representations, and from these a list of all representations
of a given degree or level may be produced. The format is designed to be
useful in the study of modular tensor categories in particular. 

 
\section{\textcolor{Chapter }{Installation}}\label{Chapter_Introduction_Section_Installation}
\logpage{[ 1, 1, 0 ]}
\hyperdef{L}{X8360C04082558A12}{}
{
  

 To install \texttt{SL2Reps}, first download it from \texttt{TODO}, then place it in the \texttt{pkg} subdirectory of your GAP installation (or in the \texttt{pkg} subdirectory of any other GAP root directory, for example one added with the \texttt{-l} argument). 

 \texttt{SL2Reps} is then loaded with the GAP command 

 \texttt{gap{\textgreater} LoadPackage( "SL2Reps" );} 

 }

 
\section{\textcolor{Chapter }{Usage}}\label{Chapter_Introduction_Section_Usage}
\logpage{[ 1, 2, 0 ]}
\hyperdef{L}{X86A9B6F87E619FFF}{}
{
  

 Specific irreducible representations may be constructed via the methods in
Chapter \ref{Chapter_Irreps}, while lists of irreducible representations with a given degree or level may
be constructed with those in Chapter \ref{Chapter_Lists}.

 This package uses an \texttt{InfoClass}, \texttt{InfoSL2Reps}. It may be set to \texttt{0} (silent), \texttt{1} (info), or \texttt{2} (verbose). To change it, use 

 \texttt{gap{\textgreater} SetInfoLevel(InfoSL2Reps, k);} 

 }

 }

   
\chapter{\textcolor{Chapter }{Description}}\label{Chapter_Description}
\logpage{[ 2, 0, 0 ]}
\hyperdef{L}{X7BBCB13F82ACC213}{}
{
  

 The group $\mathrm{SL}_2(\mathbb{Z})$ is generated by $\mathfrak{s}$ = \texttt{[[0,1],[-1,0]]} and $\mathfrak{t}$ = \texttt{[[1,1],[0,1]]} (which satisfy the relations $\mathfrak{s}^4 = (\mathfrak{st})^3 = \mathrm{id}$). Thus, any complex representation $\rho : \mathrm{SL}_2(\mathbb{Z}) \to \mathbb{C}^n$ (where $n \in \mathbb{Z}^+$ is called the \emph{degree} of $\rho$) is defined by the $n \times n$ matrices $S = \rho(\mathfrak{s})$ and $T = \rho(\mathfrak{t})$. In fact, any such representation factors through $\mathrm{SL}_2(\mathbb{Z}/\ell\mathbb{Z})$ for some $\ell \in \mathbb{Z}^+$; the smallest such $\ell$ is called the \emph{level} of $\rho$. We therefore present representations in the form of a record

 \texttt{rec(S, T, degree, level, name)} 

 where the name follows the conventions of \cite{NW76}. Note that our definition of $\mathfrak{s}$ follows that of \cite{Nobs1}; other authors prefer the inverse, i.e. \texttt{[[0,-1],[1,0]]}. When working with that convention, one must invert the $S$ matrices output by this package.

 
\section{\textcolor{Chapter }{Construction}}\label{Chapter_Description_Section_Construction}
\logpage{[ 2, 1, 0 ]}
\hyperdef{L}{X7F6278CD87400D49}{}
{
  

 For any representation $\rho$ of $\mathrm{SL}_2(\mathbb{Z})$, we may decompose $\rho$ as a direct sum of irreducible representations of prime-power level using the
Chinese remainder theorem. Therefore, to characterize all representations of $\mathrm{SL}_2(\mathbb{Z})$, it suffices to consider irreducible representations of $\mathrm{SL}_2(\mathbb{Z}/p^\lambda\mathbb{Z})$. 

 Such representations may be constructed using Weyl representations as
described in \cite[Section 1]{Nobs1}. The Weyl representations used by this package have the following form. Let $p$ be a prime number and $\lambda \in \mathbb{Z}^+$. Choose a $\mathbb{Z}/p^\lambda\mathbb{Z}$-module $M$ of rank 1 or 2, and a quadratic form $Q$ on $M$, such that $(M,Q)$ is of one of the three types described in \ref{Chapter_Description_Section_Weyl}. The \emph{quadratic module} $(M,Q)$ then gives rise to a representation of $\mathrm{SL}_2(\mathbb{Z}/p^\lambda\mathbb{Z})$, denoted $W(M,Q)$. 

 With a finite number of exceptions, every representation of $\mathrm{SL}_2(\mathbb{Z}/p^\lambda\mathbb{Z})$ may be found as a subrepresentation of $W(M,Q)$ for an appropriate choice of $(M,Q)$ \cite[Hauptsatz 2]{NW76}. The 18 exceptions may be found as the tensor product of two such
subrepresentations; these may be generated with \texttt{SL2Reps{\textunderscore}Exceptions} (\ref{SL2RepsuScoreExceptions}). 

 Representations of $\mathrm{SL}_2(\mathbb{Z})$ may then be found as direct sums of these prime-power representations. 

 }

 
\section{\textcolor{Chapter }{Weyl representation types}}\label{Chapter_Description_Section_Weyl}
\logpage{[ 2, 2, 0 ]}
\hyperdef{L}{X860C8CCC81CC703F}{}
{
  

 
\subsection{\textcolor{Chapter }{Type D}}\label{Chapter_Description_Section_Weyl_Subsection_Type_D}
\logpage{[ 2, 2, 1 ]}
\hyperdef{L}{X7AEC048F793F1D79}{}
{
  

 Let $p$ be prime and $\lambda \geq 1$. Then the Weyl representation arising from the quadratic module with $M = \mathbb{Z}/p^\lambda\mathbb{Z} \oplus \mathbb{Z}/p^\lambda\mathbb{Z}$ and $Q(x,y) = \frac{xy}{p^\lambda}$ is said to be of type $D$ and denoted $D(p,\lambda)$. Information on $(M,Q)$ may be obtained via \texttt{SL2Reps{\textunderscore}ModuleD} (\ref{SL2RepsuScoreModuleD}), and subrepresentations of $D(p,\lambda)$ with level $p^\lambda$ may be constructed via \texttt{SL2Reps{\textunderscore}RepD} (\ref{SL2RepsuScoreRepD}). 

 }

 
\subsection{\textcolor{Chapter }{Type N}}\label{Chapter_Description_Section_Weyl_Subsection_Type_N}
\logpage{[ 2, 2, 2 ]}
\hyperdef{L}{X84E15A1E8326AF4F}{}
{
  

 Let $p$ be prime and $\lambda \geq 1$. Then the Weyl representation arising from the quadratic module with $M = \mathbb{Z}/p^\lambda\mathbb{Z} \oplus \mathbb{Z}/p^\lambda\mathbb{Z}$ and $Q(x,y) = \frac{x^2 +xy+\frac{1+u}{4}y^2}{p^\lambda}$ (where, for $p \neq 2$, $u$ is chosen so that $u \equiv 3$ mod 4 with $\left(\frac{-u}{p}\right) = -1$, and for $p=2$, $u=3$) is said to be of type $N$ and denoted $N(p,\lambda)$. Information on $(M,Q)$ may be obtained via \texttt{SL2Reps{\textunderscore}ModuleN} (\ref{SL2RepsuScoreModuleN}), and subrepresentations of $D(p,\lambda)$ with level $p^\lambda$ may be constructed via \texttt{SL2Reps{\textunderscore}RepN} (\ref{SL2RepsuScoreRepN}). 

 }

 
\subsection{\textcolor{Chapter }{Type R}}\label{Chapter_Description_Section_Weyl_Subsection_Type_R}
\logpage{[ 2, 2, 3 ]}
\hyperdef{L}{X85A14FDA8467E361}{}
{
  

 The construction of type $R$ varies depending on whether $p=2$. 

 First, if $p$ is an odd prime, let $\lambda \geq 2$, $\sigma \in \{1, \dots, \lambda\}$, and $r,t \in \{1,u\}$ with $u$ a quadratic non-residue mod $p$. Then define $M = \mathbb{Z}/p^\lambda\mathbb{Z} \oplus
\mathbb{Z}/p^{\lambda-\sigma}\mathbb{Z}$ and $Q(x,y) = \frac{r(x^2 + p^\sigma t y^2)}{p^\lambda}$. 

 On the other hand, if $p=2$, let $\lambda \geq 2$, $\sigma \in \{0, \dots, \lambda-2\}$ and $r,t \in \{1,3,5,7\}$. Then define $M = \mathbb{Z}/2^{\lambda-1}\mathbb{Z} \oplus
\mathbb{Z}/2^{\lambda-\sigma-1}\mathbb{Z}$ and $Q(x,y) = \frac{r(x^2 + 2^\sigma t y^2)}{2^\lambda}$. 

 In either case, the resulting representation is said to be of type $R$ and denoted $R(p,\lambda,\sigma,r,t)$. Information on $(M,Q)$ may be obtained via \texttt{SL2Reps{\textunderscore}ModuleR} (\ref{SL2RepsuScoreModuleR}), and subrepresentations of $R(p,\lambda,\sigma,r,t)$ with level $p^\lambda$ may be constructed via \texttt{SL2Reps{\textunderscore}RepR} (\ref{SL2RepsuScoreRepR}). Note that if $\sigma = \lambda$ for $p \neq 2$, then the second factor of $M$ is trivial (and hence $t$ is irrelevant); this special case is handled by \texttt{SL2Reps{\textunderscore}RepRUnary} (\ref{SL2RepsuScoreRepRUnary}). }

 }

 }

   
\chapter{\textcolor{Chapter }{Lists of representations}}\label{Chapter_Lists}
\logpage{[ 3, 0, 0 ]}
\hyperdef{L}{X8168FF017B2C0BB2}{}
{
  
\section{\textcolor{Chapter }{Lists by degree}}\label{Chapter_Lists_Section_Degree}
\logpage{[ 3, 1, 0 ]}
\hyperdef{L}{X80406B597E11D7C6}{}
{
  

\subsection{\textcolor{Chapter }{SL2Reps{\textunderscore}PrimePowerIrrepsOfDegree}}
\logpage{[ 3, 1, 1 ]}\nobreak
\hyperdef{L}{X7F86FF007C1D95BC}{}
{\noindent\textcolor{FuncColor}{$\triangleright$\enspace\texttt{SL2Reps{\textunderscore}PrimePowerIrrepsOfDegree({\mdseries\slshape degree})\index{SL2RepsuScorePrimePowerIrrepsOfDegree@\texttt{SL2}\-\texttt{Reps{\textunderscore}}\-\texttt{Prime}\-\texttt{Power}\-\texttt{Irreps}\-\texttt{Of}\-\texttt{Degree}}
\label{SL2RepsuScorePrimePowerIrrepsOfDegree}
}\hfill{\scriptsize (function)}}\\
\textbf{\indent Returns:\ }
a list of records of the form \texttt{rec(S, T, degree, level, name)} 



 Constructs a list of all irreps of $\mathrm{SL}_2(\mathbb{Z})$ that are exactly the given degree and have prime power level. }

 

\subsection{\textcolor{Chapter }{SL2Reps{\textunderscore}PrimePowerIrrepsOfDegreeAtMost}}
\logpage{[ 3, 1, 2 ]}\nobreak
\hyperdef{L}{X7EC26C3878FA3090}{}
{\noindent\textcolor{FuncColor}{$\triangleright$\enspace\texttt{SL2Reps{\textunderscore}PrimePowerIrrepsOfDegreeAtMost({\mdseries\slshape max{\textunderscore}degree})\index{SL2RepsuScorePrimePowerIrrepsOfDegreeAtMost@\texttt{SL2}\-\texttt{Reps{\textunderscore}}\-\texttt{Prime}\-\texttt{Power}\-\texttt{Irreps}\-\texttt{Of}\-\texttt{Degree}\-\texttt{At}\-\texttt{Most}}
\label{SL2RepsuScorePrimePowerIrrepsOfDegreeAtMost}
}\hfill{\scriptsize (function)}}\\
\textbf{\indent Returns:\ }
a list of records of the form \texttt{rec(S, T, degree, level, name)} 



 Constructs a list of all irreps of $\mathrm{SL}_2(\mathbb{Z})$ that are at most the given degree and have prime power level. }

 

\subsection{\textcolor{Chapter }{SL2Reps{\textunderscore}IrrepsOfDegree}}
\logpage{[ 3, 1, 3 ]}\nobreak
\hyperdef{L}{X7E0CE1527A1F87FF}{}
{\noindent\textcolor{FuncColor}{$\triangleright$\enspace\texttt{SL2Reps{\textunderscore}IrrepsOfDegree({\mdseries\slshape degree})\index{SL2RepsuScoreIrrepsOfDegree@\texttt{SL2}\-\texttt{Reps{\textunderscore}}\-\texttt{Irreps}\-\texttt{Of}\-\texttt{Degree}}
\label{SL2RepsuScoreIrrepsOfDegree}
}\hfill{\scriptsize (function)}}\\
\textbf{\indent Returns:\ }
a list of records of the form \texttt{rec(S, T, degree, level, name)} 



 Constructs a list of all irreps of $\mathrm{SL}_2(\mathbb{Z})$ that are exactly the given degree. }

 

\subsection{\textcolor{Chapter }{SL2Reps{\textunderscore}IrrepsOfDegreeAtMost}}
\logpage{[ 3, 1, 4 ]}\nobreak
\hyperdef{L}{X84DC721D7FB5A944}{}
{\noindent\textcolor{FuncColor}{$\triangleright$\enspace\texttt{SL2Reps{\textunderscore}IrrepsOfDegreeAtMost({\mdseries\slshape degree})\index{SL2RepsuScoreIrrepsOfDegreeAtMost@\texttt{SL2}\-\texttt{Reps{\textunderscore}}\-\texttt{Irreps}\-\texttt{Of}\-\texttt{Degree}\-\texttt{At}\-\texttt{Most}}
\label{SL2RepsuScoreIrrepsOfDegreeAtMost}
}\hfill{\scriptsize (function)}}\\
\textbf{\indent Returns:\ }
a list of records of the form \texttt{rec(S, T, degree, level, name)} 



 Constructs a list of all irreps of $\mathrm{SL}_2(\mathbb{Z})$ that are at most the given degree. }

 }

 
\section{\textcolor{Chapter }{Lists by level}}\label{Chapter_Lists_Section_Level}
\logpage{[ 3, 2, 0 ]}
\hyperdef{L}{X78C583957D3FF6ED}{}
{
  

\subsection{\textcolor{Chapter }{SL2Reps{\textunderscore}PrimePowerIrrepsOfLevel}}
\logpage{[ 3, 2, 1 ]}\nobreak
\hyperdef{L}{X85C95DC07AC11205}{}
{\noindent\textcolor{FuncColor}{$\triangleright$\enspace\texttt{SL2Reps{\textunderscore}PrimePowerIrrepsOfLevel({\mdseries\slshape p, lambda})\index{SL2RepsuScorePrimePowerIrrepsOfLevel@\texttt{SL2}\-\texttt{Reps{\textunderscore}}\-\texttt{Prime}\-\texttt{Power}\-\texttt{Irreps}\-\texttt{Of}\-\texttt{Level}}
\label{SL2RepsuScorePrimePowerIrrepsOfLevel}
}\hfill{\scriptsize (function)}}\\
\textbf{\indent Returns:\ }
a list of records of the form \texttt{rec(S, T, degree, level, name)} 



 Constructs a list of all irreps of $\mathrm{SL}_2(\mathbb{Z})$ with level exactly $p^\lambda$. }

 }

 
\section{\textcolor{Chapter }{Lists of exceptional representations}}\label{Chapter_Lists_Section_Exceptions}
\logpage{[ 3, 3, 0 ]}
\hyperdef{L}{X80400C6D79D4D0D6}{}
{
  

\subsection{\textcolor{Chapter }{SL2Reps{\textunderscore}Exceptions}}
\logpage{[ 3, 3, 1 ]}\nobreak
\hyperdef{L}{X82646BF88466E216}{}
{\noindent\textcolor{FuncColor}{$\triangleright$\enspace\texttt{SL2Reps{\textunderscore}Exceptions({\mdseries\slshape arg})\index{SL2RepsuScoreExceptions@\texttt{SL2Reps{\textunderscore}Exceptions}}
\label{SL2RepsuScoreExceptions}
}\hfill{\scriptsize (function)}}\\
\textbf{\indent Returns:\ }
a list of records of the form \texttt{rec(S, T, degree, level, name)} 



 Constructs a list of the 18 exceptional irreps of $\mathrm{SL}_2(\mathbb{Z})$. }

 }

 }

   
\chapter{\textcolor{Chapter }{Methods for testing}}\label{Chapter_Testing}
\logpage{[ 4, 0, 0 ]}
\hyperdef{L}{X794C1A137F8FA14D}{}
{
  
\section{\textcolor{Chapter }{Testing}}\label{Chapter_Testing_Section_Testing}
\logpage{[ 4, 1, 0 ]}
\hyperdef{L}{X7DE7E7187BE24368}{}
{
  

\subsection{\textcolor{Chapter }{SL2Reps{\textunderscore}SL2Conj}}
\logpage{[ 4, 1, 1 ]}\nobreak
\hyperdef{L}{X7F816DE77BADF5BC}{}
{\noindent\textcolor{FuncColor}{$\triangleright$\enspace\texttt{SL2Reps{\textunderscore}SL2Conj({\mdseries\slshape p, ld})\index{SL2RepsuScoreSL2Conj@\texttt{SL2Reps{\textunderscore}SL2Conj}}
\label{SL2RepsuScoreSL2Conj}
}\hfill{\scriptsize (function)}}\\
\textbf{\indent Returns:\ }
the group $\mathrm{SL}_2(\mathbb{Z}/p^\lambda\mathbb{Z})$ with conjugacy classes set to the format we use. 



 

 }

 

\subsection{\textcolor{Chapter }{SL2Reps{\textunderscore}ChiST}}
\logpage{[ 4, 1, 2 ]}\nobreak
\hyperdef{L}{X78E0537D7C06CAEE}{}
{\noindent\textcolor{FuncColor}{$\triangleright$\enspace\texttt{SL2Reps{\textunderscore}ChiST({\mdseries\slshape S, T, p, ld})\index{SL2RepsuScoreChiST@\texttt{SL2Reps{\textunderscore}ChiST}}
\label{SL2RepsuScoreChiST}
}\hfill{\scriptsize (function)}}\\
\textbf{\indent Returns:\ }
a list representing a character of $\mathrm{SL}_2(\mathbb{Z}/p^\lambda\mathbb{Z})$ 



 Converts the modular data $(S,T)$, which must have level dividing $p^\lambda$, into a character of $\mathrm{SL}_2(\mathbb{Z}/p^\lambda\mathbb{Z})$, presented in a form matching the conjugacy classes used in \texttt{SL2Reps{\textunderscore}SL2Conj}. }

 

\subsection{\textcolor{Chapter }{SL2Reps{\textunderscore}IrrepPositionTest}}
\logpage{[ 4, 1, 3 ]}\nobreak
\hyperdef{L}{X7E8C59607A1226E7}{}
{\noindent\textcolor{FuncColor}{$\triangleright$\enspace\texttt{SL2Reps{\textunderscore}IrrepPositionTest({\mdseries\slshape p, lambda})\index{SL2RepsuScoreIrrepPositionTest@\texttt{SL2}\-\texttt{Reps{\textunderscore}}\-\texttt{Irrep}\-\texttt{Position}\-\texttt{Test}}
\label{SL2RepsuScoreIrrepPositionTest}
}\hfill{\scriptsize (function)}}\\
\textbf{\indent Returns:\ }
a boolean 



 Constructs and tests all irreps of level dividing $p^\lambda$ by checking their positions in \texttt{Irr(G)}. }

 }

 }

   
\chapter{\textcolor{Chapter }{Irreducible representations of prime-power level}}\label{Chapter_Irreps}
\logpage{[ 5, 0, 0 ]}
\hyperdef{L}{X7C4165447B8AB223}{}
{
  
\section{\textcolor{Chapter }{Representations of type D}}\label{Chapter_Irreps_Section_Representations_of_type_D}
\logpage{[ 5, 1, 0 ]}
\hyperdef{L}{X82DA126D7F755B1C}{}
{
  

 See \ref{Chapter_Description_Section_Weyl_Subsection_Type_D}. 

\subsection{\textcolor{Chapter }{SL2Reps{\textunderscore}ModuleD}}
\logpage{[ 5, 1, 1 ]}\nobreak
\hyperdef{L}{X78D4833F85945690}{}
{\noindent\textcolor{FuncColor}{$\triangleright$\enspace\texttt{SL2Reps{\textunderscore}ModuleD({\mdseries\slshape p, ld})\index{SL2RepsuScoreModuleD@\texttt{SL2Reps{\textunderscore}ModuleD}}
\label{SL2RepsuScoreModuleD}
}\hfill{\scriptsize (function)}}\\
\textbf{\indent Returns:\ }
a record \texttt{rec(Agrp, Bp, Char, IsPrim)} describing $(M,Q)$



 Constructs information about the underlying quadratic module $(M,Q)$ of type $D$.

 \texttt{Agrp} describes the elements of $\mathfrak{A} = (\mathbb{Z}/p^\lambda\mathbb{Z})^\times$ (see \cite[Section 2.1]{NW76}).

 \texttt{Bp} describes a set of representatives for the $\mathfrak{A}$-orbits on $M^\times$, which correspond to a basis the $\mathrm{SL}_2(\mathbb{Z}/p^\lambda\mathbb{Z})$-invariant subspace associated to any primitive character $\chi \in \hat{\mathfrak{A}}$ with $\chi^2 \not\equiv 1$. For other characters, we must use different bases which are particular to
each case.

 \texttt{Char(i,j)} converts the \texttt{chi{\textunderscore}index} used in \texttt{SL2Reps{\textunderscore}RepD} (\ref{SL2RepsuScoreRepD}) to a function.

 \texttt{IsPrim(chi)} tests whether a given character (e.g. from \texttt{Char(i,j)}) is primitive. }

 

\subsection{\textcolor{Chapter }{SL2Reps{\textunderscore}RepD}}
\logpage{[ 5, 1, 2 ]}\nobreak
\hyperdef{L}{X84376A29780650AD}{}
{\noindent\textcolor{FuncColor}{$\triangleright$\enspace\texttt{SL2Reps{\textunderscore}RepD({\mdseries\slshape p, ld, chi{\textunderscore}index})\index{SL2RepsuScoreRepD@\texttt{SL2Reps{\textunderscore}RepD}}
\label{SL2RepsuScoreRepD}
}\hfill{\scriptsize (function)}}\\
\textbf{\indent Returns:\ }
a list of lists of the form $[S,T]$ 



 Constructs the modular data for the irreducible representation(s) of type $D$ with level $p^\lambda$ corresponding to the character $\chi$ indexed by \texttt{chi{\textunderscore}index}. }

 }

 
\section{\textcolor{Chapter }{Representations of type N}}\label{Chapter_Irreps_Section_Representations_of_type_N}
\logpage{[ 5, 2, 0 ]}
\hyperdef{L}{X7CD74CFC84B38042}{}
{
  

 See \ref{Chapter_Description_Section_Weyl_Subsection_Type_N}. 

\subsection{\textcolor{Chapter }{SL2Reps{\textunderscore}ModuleN}}
\logpage{[ 5, 2, 1 ]}\nobreak
\hyperdef{L}{X86D9DDAE83E4A141}{}
{\noindent\textcolor{FuncColor}{$\triangleright$\enspace\texttt{SL2Reps{\textunderscore}ModuleN({\mdseries\slshape p, ld})\index{SL2RepsuScoreModuleN@\texttt{SL2Reps{\textunderscore}ModuleN}}
\label{SL2RepsuScoreModuleN}
}\hfill{\scriptsize (function)}}\\
\textbf{\indent Returns:\ }
a record \texttt{rec(Agrp, Bp, Char, Nm, Prod)} describing $(M,Q)$



 Constructs information about the underlying quadratic module $(M,Q)$ of type $N$.

 \texttt{Agrp} describes the elements of $\mathfrak{A} = \{\varepsilon \in M^\times \mid \operatorname{Nm}(\varepsilon)
= 1 \}$ (see \cite[Section 2.2]{NW76}).

 \texttt{Bp} describes a set of representatives for the $\mathfrak{A}$-orbits on $M^\times$, which correspond to a basis the $\mathrm{SL}_2(\mathbb{Z}/p^\lambda\mathbb{Z})$-invariant subspace associated to any primitive character $\chi \in \hat{\mathfrak{A}}$ with $\chi^2 \not\equiv 1$. For other characters, we must use different bases which are particular to
each case.

 \texttt{Char(i,j)} converts the \texttt{chi{\textunderscore}index} used in \texttt{SL2Reps{\textunderscore}RepN} (\ref{SL2RepsuScoreRepN}) to a function.

 \texttt{Nm(a)} and \texttt{Prod(a,b)} are the norm and product functions on $M$, respectively. }

 

\subsection{\textcolor{Chapter }{SL2Reps{\textunderscore}RepN}}
\logpage{[ 5, 2, 2 ]}\nobreak
\hyperdef{L}{X7A3A34B886A52E2B}{}
{\noindent\textcolor{FuncColor}{$\triangleright$\enspace\texttt{SL2Reps{\textunderscore}RepN({\mdseries\slshape p, ld, chi{\textunderscore}index})\index{SL2RepsuScoreRepN@\texttt{SL2Reps{\textunderscore}RepN}}
\label{SL2RepsuScoreRepN}
}\hfill{\scriptsize (function)}}\\
\textbf{\indent Returns:\ }
a list of lists of the form $[S,T]$ 



 Constructs the modular data for the irreducible representation(s) of type $N$ with level $p^\lambda$ corresponding to the character $\chi$ indexed by \texttt{chi{\textunderscore}index}. }

 }

 
\section{\textcolor{Chapter }{Representations of type R}}\label{Chapter_Irreps_Section_Representations_of_type_R}
\logpage{[ 5, 3, 0 ]}
\hyperdef{L}{X7D9759387C499FA0}{}
{
  

 See \ref{Chapter_Description_Section_Weyl_Subsection_Type_R}. 

\subsection{\textcolor{Chapter }{SL2Reps{\textunderscore}ModuleR}}
\logpage{[ 5, 3, 1 ]}\nobreak
\hyperdef{L}{X8799C86A83FDD22B}{}
{\noindent\textcolor{FuncColor}{$\triangleright$\enspace\texttt{SL2Reps{\textunderscore}ModuleR({\mdseries\slshape p, ld, sigma, r, t})\index{SL2RepsuScoreModuleR@\texttt{SL2Reps{\textunderscore}ModuleR}}
\label{SL2RepsuScoreModuleR}
}\hfill{\scriptsize (function)}}\\
\textbf{\indent Returns:\ }
a record \texttt{rec(Agrp, Char, IsPrim, Nm, Ord, Prod, c, tM)} describing $(M,Q)$



 Constructs information about the underlying quadratic module $(M,Q)$ of type $R$.

 \texttt{Agrp} describes the elements of $\mathfrak{A} = \{\varepsilon \in M^\times \mid \operatorname{Nm}(\varepsilon)
= 1 \}$ (see \cite[Section 2.3 - 2.5]{NW76}).

 Representatives for the $\mathfrak{A}$-orbits on $M^\times$ can depend on the choice of character, even for primitive characters $\chi$ with $\chi^2 \not\equiv 1$. Thus, we cannot provide them here, and they are instead calculated by \texttt{SL2Reps{\textunderscore}RepR} (\ref{SL2RepsuScoreRepR}).

 \texttt{Char(i,j)} converts the \texttt{chi{\textunderscore}index} used in \texttt{SL2Reps{\textunderscore}RepR} (\ref{SL2RepsuScoreRepR}) to a function.

 \texttt{IsPrim(chi)} tests whether a given character (e.g. from \texttt{Char}) is primitive.

 \texttt{Nm(a)}, \texttt{Ord(a)}, and \texttt{Prod(a,b)} are the norm, order, and product functions on $M$, respectively.

 \texttt{c} is a scalar used in calculating the $S$-matrix; namely $c = \frac{1}{|M|} \sum_{x \in M} \mathbf{e}(Q(x))$.

 \texttt{tM} is the group $M - pM$. }

 

\subsection{\textcolor{Chapter }{SL2Reps{\textunderscore}RepR}}
\logpage{[ 5, 3, 2 ]}\nobreak
\hyperdef{L}{X7B7A217C8634042C}{}
{\noindent\textcolor{FuncColor}{$\triangleright$\enspace\texttt{SL2Reps{\textunderscore}RepR({\mdseries\slshape p, ld, sigma, r, t, chi{\textunderscore}index})\index{SL2RepsuScoreRepR@\texttt{SL2Reps{\textunderscore}RepR}}
\label{SL2RepsuScoreRepR}
}\hfill{\scriptsize (function)}}\\
\textbf{\indent Returns:\ }
a list of lists of the form $[S,T]$ 



 Constructs the modular data for the irreducible representation(s) of type $R$ with level $p^\lambda$ corresponding to the character $\chi$ indexed by \texttt{chi{\textunderscore}index}.

 When $\sigma = \lambda$, this falls through to \texttt{SL2Reps{\textunderscore}RepRUnary} (\ref{SL2RepsuScoreRepRUnary}). }

 

\subsection{\textcolor{Chapter }{SL2Reps{\textunderscore}RepRUnary}}
\logpage{[ 5, 3, 3 ]}\nobreak
\hyperdef{L}{X87D9D1497ACD4A1E}{}
{\noindent\textcolor{FuncColor}{$\triangleright$\enspace\texttt{SL2Reps{\textunderscore}RepRUnary({\mdseries\slshape p, ld, r})\index{SL2RepsuScoreRepRUnary@\texttt{SL2Reps{\textunderscore}RepRUnary}}
\label{SL2RepsuScoreRepRUnary}
}\hfill{\scriptsize (function)}}\\
\textbf{\indent Returns:\ }
a list of lists of the form $[S,T]$ 



 Constructs the modular data for the irreducible representation(s) of unary
type $R$ (that is, with $\sigma = \lambda$) with level $p^\lambda$. }

 }

 }

 \def\bibname{References\logpage{[ "Bib", 0, 0 ]}
\hyperdef{L}{X7A6F98FD85F02BFE}{}
}

\bibliographystyle{alpha}
\bibliography{SL2Reps.bib}

\addcontentsline{toc}{chapter}{References}

\def\indexname{Index\logpage{[ "Ind", 0, 0 ]}
\hyperdef{L}{X83A0356F839C696F}{}
}

\cleardoublepage
\phantomsection
\addcontentsline{toc}{chapter}{Index}


\printindex

\immediate\write\pagenrlog{["Ind", 0, 0], \arabic{page},}
\immediate\write\pagenrlog{["Ind", 0, 0], \arabic{page},}
\immediate\write\pagenrlog{["Ind", 0, 0], \arabic{page},}
\immediate\write\pagenrlog{["Ind", 0, 0], \arabic{page},}
\immediate\write\pagenrlog{["Ind", 0, 0], \arabic{page},}
\immediate\write\pagenrlog{["Ind", 0, 0], \arabic{page},}
\immediate\write\pagenrlog{["Ind", 0, 0], \arabic{page},}
\immediate\write\pagenrlog{["Ind", 0, 0], \arabic{page},}
\immediate\write\pagenrlog{["Ind", 0, 0], \arabic{page},}
\immediate\write\pagenrlog{["Ind", 0, 0], \arabic{page},}
\newpage
\immediate\write\pagenrlog{["End"], \arabic{page}];}
\immediate\closeout\pagenrlog
\end{document}
