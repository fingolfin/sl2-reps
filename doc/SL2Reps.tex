% generated by GAPDoc2LaTeX from XML source (Frank Luebeck)
\documentclass[a4paper,11pt]{report}

\usepackage[top=37mm,bottom=37mm,left=27mm,right=27mm]{geometry}
\sloppy
\pagestyle{myheadings}
\usepackage{amssymb,amsmath}
\usepackage[utf8]{inputenc}
\usepackage{makeidx}
\makeindex
\usepackage{color}
\definecolor{FireBrick}{rgb}{0.5812,0.0074,0.0083}
\definecolor{RoyalBlue}{rgb}{0.0236,0.0894,0.6179}
\definecolor{RoyalGreen}{rgb}{0.0236,0.6179,0.0894}
\definecolor{RoyalRed}{rgb}{0.6179,0.0236,0.0894}
\definecolor{LightBlue}{rgb}{0.8544,0.9511,1.0000}
\definecolor{Black}{rgb}{0.0,0.0,0.0}

\definecolor{linkColor}{rgb}{0.0,0.0,0.554}
\definecolor{citeColor}{rgb}{0.0,0.0,0.554}
\definecolor{fileColor}{rgb}{0.0,0.0,0.554}
\definecolor{urlColor}{rgb}{0.0,0.0,0.554}
\definecolor{promptColor}{rgb}{0.0,0.0,0.589}
\definecolor{brkpromptColor}{rgb}{0.589,0.0,0.0}
\definecolor{gapinputColor}{rgb}{0.589,0.0,0.0}
\definecolor{gapoutputColor}{rgb}{0.0,0.0,0.0}

%%  for a long time these were red and blue by default,
%%  now black, but keep variables to overwrite
\definecolor{FuncColor}{rgb}{0.0,0.0,0.0}
%% strange name because of pdflatex bug:
\definecolor{Chapter }{rgb}{0.0,0.0,0.0}
\definecolor{DarkOlive}{rgb}{0.1047,0.2412,0.0064}


\usepackage{fancyvrb}

\usepackage{mathptmx,helvet}
\usepackage[T1]{fontenc}
\usepackage{textcomp}


\usepackage[
            pdftex=true,
            bookmarks=true,        
            a4paper=true,
            pdftitle={Written with GAPDoc},
            pdfcreator={LaTeX with hyperref package / GAPDoc},
            colorlinks=true,
            backref=page,
            breaklinks=true,
            linkcolor=linkColor,
            citecolor=citeColor,
            filecolor=fileColor,
            urlcolor=urlColor,
            pdfpagemode={UseNone}, 
           ]{hyperref}

\newcommand{\maintitlesize}{\fontsize{50}{55}\selectfont}

% write page numbers to a .pnr log file for online help
\newwrite\pagenrlog
\immediate\openout\pagenrlog =\jobname.pnr
\immediate\write\pagenrlog{PAGENRS := [}
\newcommand{\logpage}[1]{\protect\write\pagenrlog{#1, \thepage,}}
%% were never documented, give conflicts with some additional packages

\newcommand{\GAP}{\textsf{GAP}}

%% nicer description environments, allows long labels
\usepackage{enumitem}
\setdescription{style=nextline}

%% depth of toc
\setcounter{tocdepth}{1}





%% command for ColorPrompt style examples
\newcommand{\gapprompt}[1]{\color{promptColor}{\bfseries #1}}
\newcommand{\gapbrkprompt}[1]{\color{brkpromptColor}{\bfseries #1}}
\newcommand{\gapinput}[1]{\color{gapinputColor}{#1}}


\begin{document}

\logpage{[ 0, 0, 0 ]}
\begin{titlepage}
\mbox{}\vfill

\begin{center}{\maintitlesize \textbf{ SL2Reps \mbox{}}}\\
\vfill

\hypersetup{pdftitle= SL2Reps }
\markright{\scriptsize \mbox{}\hfill  SL2Reps  \hfill\mbox{}}
{\Huge \textbf{ Constructs congruent representations of SL(2,Z). \mbox{}}}\\
\vfill

{\Huge  0.1 \mbox{}}\\[1cm]
{ 12 November 2021 \mbox{}}\\[1cm]
\mbox{}\\[2cm]
{\Large \textbf{ Siu-Hung Ng\\
   \mbox{}}}\\
{\Large \textbf{ Yilong Wang\\
   \mbox{}}}\\
{\Large \textbf{ Samuel Wilson\\
  \mbox{}}}\\
\hypersetup{pdfauthor= Siu-Hung Ng\\
   ;  Yilong Wang\\
   ;  Samuel Wilson\\
  }
\end{center}\vfill

\mbox{}\\
{\mbox{}\\
\small \noindent \textbf{ Siu-Hung Ng\\
   }  Email: \href{mailto://rng@math.lsu.edu} {\texttt{rng@math.lsu.edu}}\\
  Homepage: \href{https://www.math.lsu.edu/~rng/} {\texttt{https://www.math.lsu.edu/\texttt{\symbol{126}}rng/}}}\\
{\mbox{}\\
\small \noindent \textbf{ Yilong Wang\\
   }  Email: \href{mailto://wyl@bimsa.cn} {\texttt{wyl@bimsa.cn}}\\
  Homepage: \href{https://yilongwang11.github.io} {\texttt{https://yilongwang11.github.io}}}\\
{\mbox{}\\
\small \noindent \textbf{ Samuel Wilson\\
  }  Email: \href{mailto://swil311@lsu.edu} {\texttt{swil311@lsu.edu}}}\\
\end{titlepage}

\newpage\setcounter{page}{2}
{\small 
\section*{Copyright}
\logpage{[ 0, 0, 1 ]}
 \index{License} {\copyright} 2021 by Siu-Hung Ng, Yilong Wang, and Samuel Wilson

 This package is free software; you can redistribute it and/or modify it under
the terms of the \href{http://www.fsf.org/licenses/gpl.html} {GNU General Public License} as published by the Free Software Foundation; either version 2 of the License,
or (at your option) any later version. \mbox{}}\\[1cm]
{\small 
\section*{Acknowledgements}
\logpage{[ 0, 0, 2 ]}
 This project is partially supported by NSF grant DMS 1664418. \mbox{}}\\[1cm]
\newpage

\def\contentsname{Contents\logpage{[ 0, 0, 3 ]}}

\tableofcontents
\newpage

     
\chapter{\textcolor{Chapter }{Introduction}}\label{Chapter_Introduction}
\logpage{[ 1, 0, 0 ]}
\hyperdef{L}{X7DFB63A97E67C0A1}{}
{
  

 This package, \texttt{SL2Reps}, provides methods for constructing and testing matrix presentations of the
representations of $\mathrm{SL}_2(\mathbb{Z})$ whose kernels are congruence subgroups of $\mathrm{SL}_2(\mathbb{Z})$. 

 Irreducible representations of prime-power level are constructed individually
by using the Weil representations of quadratic modules, and from these a list
of all representations of a given degree or level can be produced. The format
is designed for the study of modular tensor categories in particular. 

 
\section{\textcolor{Chapter }{Installation}}\label{Chapter_Introduction_Section_Installation}
\logpage{[ 1, 1, 0 ]}
\hyperdef{L}{X8360C04082558A12}{}
{
  

 To install \texttt{SL2Reps}, first download it from \texttt{https://github.com/ontoclasm/sl2-reps}, then place it in the \texttt{pkg} subdirectory of your GAP installation (or in the \texttt{pkg} subdirectory of any other GAP root directory, for example one added with the \texttt{-l} argument). 

 \texttt{SL2Reps} is then loaded with the GAP command 

 \texttt{gap{\textgreater} LoadPackage( "SL2Reps" );} 

 }

 
\section{\textcolor{Chapter }{Usage}}\label{Chapter_Introduction_Section_Usage}
\logpage{[ 1, 2, 0 ]}
\hyperdef{L}{X86A9B6F87E619FFF}{}
{
  

 Specific irreducible representations may be constructed via the methods in
Chapter \ref{Chapter_Irreps}, while lists of irreducible representations with a given degree or level may
be constructed with those in Chapter \ref{Chapter_Lists}. 

 This package uses an \texttt{InfoClass}, \texttt{InfoSL2Reps}. It may be set to \texttt{0} (silent), \texttt{1} (info), or \texttt{2} (verbose). To change it, use 

 \texttt{gap{\textgreater} SetInfoLevel( InfoSL2Reps, k );} 

 }

 }

   
\chapter{\textcolor{Chapter }{Description}}\label{Chapter_Description}
\logpage{[ 2, 0, 0 ]}
\hyperdef{L}{X7BBCB13F82ACC213}{}
{
  

 The group $\mathrm{SL}_2(\mathbb{Z})$ is generated by $\mathfrak{s}$ = \texttt{[[0,1],[-1,0]]} and $\mathfrak{t}$ = \texttt{[[1,1],[0,1]]} (which satisfy the relations $\mathfrak{s}^4 = (\mathfrak{st})^3 = \mathrm{id}$). Thus, any complex representation $\rho$ of $\mathrm{SL}_2(\mathbb{Z})$ on $\mathbb{C}^n$ (where $n \in \mathbb{Z}^+$ is called the \emph{degree} of $\rho$) is determined by the $n \times n$ matrices $S = \rho(\mathfrak{s})$ and $T = \rho(\mathfrak{t})$. 

 This package constructs irreducible representations of $\mathrm{SL}_2(\mathbb{Z})$ which factor through $\mathrm{SL}_2(\mathbb{Z}/\ell\mathbb{Z})$ for some $\ell \in \mathbb{Z}^+$; the smallest such $\ell$ is called the \emph{level} of the representation. One may equivalently say that the kernel of the
representation is a congruence subgroup. It has been shown that any
representation of $\mathrm{SL}_2(\mathbb{Z})$ arising from a modular tensor category has this property \cite{DLN15}. 

 We therefore present representations in the form of a record \texttt{rec(S, T, degree, level, name)}, where the name follows the conventions of \cite{NW76}. 

 Note that our definition of $\mathfrak{s}$ follows that of \cite{Nobs1}; other authors prefer the inverse, i.e. $\mathfrak{s}$ = \texttt{[[0,-1],[1,0]]} (under which convention the relations are $\mathfrak{s}^4 = \mathrm{id}$, $(\mathfrak{s}\mathfrak{t})^3 = \mathfrak{s}^2$). When working with that convention, one must invert the $S$ matrices output by this package. 

 Throughout, we denote by $\mathbf{e}$ the map $k \mapsto e^{2 \pi i k}$ (an isomorphism from $\mathbb{Q}/\mathbb{Z}$ to the group of finite roots of unity in $\mathbb{C}$). For a group $G$, we denote by $\widehat{G}$ the character group $\operatorname{Hom}(G, \mathbb{C}^\times)$. 

 
\section{\textcolor{Chapter }{Construction}}\label{Chapter_Description_Section_Construction}
\logpage{[ 2, 1, 0 ]}
\hyperdef{L}{X7F6278CD87400D49}{}
{
  

 Any representation $\rho$ of $\mathrm{SL}_2(\mathbb{Z})$ can be decomposed into a direct sum of irreducible representations (irreps).
Further, if $\rho$ has finite level, each irrep can be factorized into a tensor product of irreps
whose levels are powers of distinct primes (using the Chinese remainder
theorem). Therefore, to characterize all finite-dimensional representations of $\mathrm{SL}_2(\mathbb{Z})$ of finite level, it suffices to consider irreps of $\mathrm{SL}_2(\mathbb{Z}/p^\lambda\mathbb{Z})$ for primes $p$ and positive integers $\lambda$. 

 
\subsection{\textcolor{Chapter }{Weil representations}}\label{Chapter_Description_Section_Construction_Subsection_Weil_representations}
\logpage{[ 2, 1, 1 ]}
\hyperdef{L}{X86466B2786DD47C4}{}
{
  

 Such representations may be constructed using Weil representations as
described in \cite[Section 1]{Nobs1}. We give a brief summary of the process here. First, if $M$ is any additive abelian group, a \emph{quadratic form} on $M$ is a map $Q : M \to \mathbb{Q}/\mathbb{Z}$ such that 
\begin{itemize}
\item $Q(-x) = Q(x)$ for all $x \in M$, and
\item $B(x,y) = Q(x+y) - Q(x) - Q(y)$ defines a $\mathbb{Z}$-bilinear map $M \times M \to \mathbb{Q}/\mathbb{Z}$.
\end{itemize}
 

 Now let $p$ be a prime number and $\lambda \in \mathbb{Z}^+$. Choose a $\mathbb{Z}/p^\lambda\mathbb{Z}$-module $M$ and a quadratic form $Q$ on $M$ such that the pair $(M,Q)$ is of one of the three types described in Section \ref{Chapter_Description_Section_Weil}. Each such $M$ is a ring, and has at most 2 cyclic factors as an additive group. Those with 2
cyclic factors may be identified with a quotient of the quadratic integers,
giving a norm on $M$. Then the \emph{quadratic module} $(M,Q)$ gives rise to a representation of $\mathrm{SL}_2(\mathbb{Z}/p^\lambda\mathbb{Z})$ on the vector space $V = \mathbb{C}^M$ of complex-valued functions on $M$. This representation is denoted $W(M,Q)$. Note that the \emph{central charge} of $(M,Q)$ is given by $S_Q(-1) = \frac{1}{\sqrt{|M|}} \sum_{x \in M} \mathbf{e}(Q(x))$. 

 We may construct subrepresentations $W(M,Q,\chi)$ of $W(M,Q)$ as follows. Denote 
\[\operatorname{Aut}(M,Q) = \{ \varepsilon \in \operatorname{Aut}(M) \mid
Q(\varepsilon x) = Q(x) \text{ for all } x \in M\}~.\]
 We then associate to $(M,Q)$ an abelian subgroup $\mathfrak{A} \leq \operatorname{Aut}(M,Q)$; the structure of this group depends on $(M,Q)$ and is described in Section \ref{Chapter_Description_Section_Weil}. Note that $\mathfrak{A}$ has at most two cyclic factors, whose generators we denote by $\alpha$ and $\beta$. Now, let $\chi \in \widehat{\mathfrak{A}}$ be a 1-dimensional representation (\emph{character}) of $\mathfrak{A}$, and define 
\[V_\chi = \{f \in V \mid f(\varepsilon x) = \chi(\varepsilon) f(x) \text{ for
all } x \in M \text{ and } \varepsilon \in \mathfrak{A}\}~,\]
 which is a $SL_2(\mathbb{Z}/p^\lambda\mathbb{Z})$-invariant subspace of $V$. We then denote by $W(M,Q,\chi)$ the subrepresentation of $W(M,Q)$ on $V_\chi$. Note that $W(M,Q,\chi) \cong W(M,Q,\overline{\chi})$. 

 }

 
\subsection{\textcolor{Chapter }{Primitive characters}}\label{Chapter_Description_Section_Construction_Subsection_Primitive_characters}
\logpage{[ 2, 1, 2 ]}
\hyperdef{L}{X828CAE1686C1C1A5}{}
{
  

 For the abelian groups $\mathfrak{A} \leq \operatorname{Aut}(M,Q)$, we will frequently refer to a character $\chi \in \widehat{\mathfrak{A}}$ as being \emph{primitive}. With the exception of a single family of modules of type $R$ (the \emph{extremal} case, for which see Section \ref{Chapter_Description_Section_Weil_Subsection_Type_R_Special}), primitivity amounts to the following: there exists some $\varepsilon \in \mathfrak{A}$ such that $\chi(\varepsilon) \neq 1$ and $\varepsilon$ fixes the submodule $pM \subset M$ pointwise. There exists a subgroup $\mathfrak{A}_0 \leq \mathfrak{A}$ such that a non-trivial $\chi \in \widehat{\mathfrak{A}}$ is primitive if and only if $\chi$ is injective on $\mathfrak{A}_0$ (or, equivalently, if $\mathfrak{A}_0 \cap \operatorname{ker} \chi$ is trivial). 

 Explicit descriptions of the group $\mathfrak{A}_0$ for each type are given in Section \ref{Chapter_Description_Section_Weil} and may be used to determine the primitive characters. 

 }

 
\subsection{\textcolor{Chapter }{Irrep types}}\label{Chapter_Description_Section_Construction_Subsection_Irrep_types}
\logpage{[ 2, 1, 3 ]}
\hyperdef{L}{X85605F1983AA938D}{}
{
  

 The prime-power irreps then fall into three cases. 
\begin{itemize}
\item The overwhelming majority are of the form $W(M,Q,\chi)$ for $\chi$ primitive and $\chi^2 \neq 1$; we call these \emph{standard}. This includes the primitive characters from the extremal case.
\item A finite number, and a single infinite family arising from the extremal case
(Section \ref{Chapter_Description_Section_Weil_Subsection_Type_R_Special}), are instead constructed by using non-primitive characters or primitive
characters $\chi$ with $\chi^2 = 1$. We call these \emph{non-standard}.
\item Finally, 18 \emph{exceptional} irreps are constructed as tensor products of two irreps from the other two
cases.
\end{itemize}
 

 All the finite-dimensional irreducible representations of $\mathrm{SL}_2(\mathbb{Z})$ of finite level can now be constructed by taking tensor products of these
prime-power irreps. Note that, if two representations are determined by pairs \texttt{[S1,T1]} and \texttt{[S2,T2]}, then the pair for their tensor product may be calculated via the GAP command \texttt{KroneckerProduct}, namely as \texttt{[KroneckerProduct(S1,S2),KroneckerProduct(T1,T2)]}. 

 }

 }

 
\section{\textcolor{Chapter }{Weil representation types}}\label{Chapter_Description_Section_Weil}
\logpage{[ 2, 2, 0 ]}
\hyperdef{L}{X861BA4A8800E1A08}{}
{
  

 
\subsection{\textcolor{Chapter }{Type D}}\label{Chapter_Description_Section_Weil_Subsection_Type_D}
\logpage{[ 2, 2, 1 ]}
\hyperdef{L}{X7AEC048F793F1D79}{}
{
  

 Let $p$ be prime. If $p=2$ or $p=3$, let $\lambda \geq 2$; otherwise, let $\lambda \geq 1$. Then the Weil representation arising from the quadratic module with 
\[M = \mathbb{Z}/p^\lambda\mathbb{Z} \oplus \mathbb{Z}/p^\lambda\mathbb{Z}
\qquad \text{and} \qquad Q(x,y) = \frac{xy}{p^\lambda}\]
 is said to be of type $D$ and denoted $D(p,\lambda)$. Information on type $D$ quadratic modules may be obtained via \texttt{SL2ModuleD} (\ref{SL2ModuleD}), and subrepresentations of $D(p,\lambda)$ with level $p^\lambda$ may be constructed via \texttt{SL2IrrepD} (\ref{SL2IrrepD}). 

 The group 
\[\mathfrak{A} \cong (\mathbb{Z}/p^\lambda\mathbb{Z})^\times\]
 acts on $M$ by $a(x,y) = (a^{-1}x, ay)$ and is thus identified with a subgroup of $\operatorname{Aut}(M,Q)$; see \cite[Section 2.1]{NW76}. The group $\mathfrak{A}$ has order $p^{\lambda-1}(p-1)$ and $\mathfrak{A} = \langle\alpha\rangle \times \langle\beta\rangle$. The relevant information for type $D$ quadratic modules is as follows: \begin{center}
\begin{tabular}{ccccc}$p$&
$\lambda$&
$\alpha$&
$\beta$&
$\mathfrak{A}_0$\\
\hline
$>2$&
$1$&
$1$&
$|\beta| = p-1$&
$\langle 1 \rangle$\\
$>2$&
$>1$&
$|\alpha| = p^{\lambda-1}$ (e.g. $\alpha = 1 + p$)&
$|\beta| = p-1$&
$\langle \alpha \rangle$\\
$2$&
$2$&
$1$&
$-1$&
$\langle 1 \rangle$\\
$2$&
$>2$&
$|\alpha| = 2^{\lambda-2}$ (e.g. $\alpha = 5$)&
$-1$&
$\langle \alpha \rangle$\\
\end{tabular}\\[2mm]
\end{center}

 When $\mathfrak{A}_0$ is trivial, every non-trivial character $\chi \in \widehat{\mathfrak{A}}$ is primitive. 

 }

 
\subsection{\textcolor{Chapter }{Type N}}\label{Chapter_Description_Section_Weil_Subsection_Type_N}
\logpage{[ 2, 2, 2 ]}
\hyperdef{L}{X84E15A1E8326AF4F}{}
{
  

 Let $p$ be prime and $\lambda \geq 1$. If $p \neq 2$, let $u$ be a positive integer so that $u \equiv 3$ mod 4 with $-u$ a quadratic non-residue mod $p$; if $p = 2$, let $u=3$. Then the Weil representation arising from the quadratic module with 
\[M = \mathbb{Z}/p^\lambda\mathbb{Z} \oplus \mathbb{Z}/p^\lambda\mathbb{Z}
\qquad \text{and} \qquad Q(x,y) = \frac{x^2 +xy+\frac{1+u}{4}y^2}{p^\lambda}\]
 is said to be of type $N$ and denoted $N(p,\lambda)$. Information on type $N$ quadratic modules may be obtained via \texttt{SL2ModuleN} (\ref{SL2ModuleN}), and subrepresentations of $N(p,\lambda)$ with level $p^\lambda$ may be constructed via \texttt{SL2IrrepN} (\ref{SL2IrrepN}). 

 The additive group $M$ is a ring with multiplication given by 
\[(x_1, y_1) \cdot (x_2, y_2) = (x_1x_2 - \frac{1+u}{4}y_1y_2, x_1y_2 + x_2y_1 +
y_1y_2)\]
 and identity element $(1,0)$. We define a norm $\operatorname{Nm}(x,y) = x^2 + xy + \frac{1+u}{4}y^2$ on $M$; then the multiplicative subgroup 
\[\mathfrak{A} = \{\varepsilon \in M^\times \mid \operatorname{Nm}(\varepsilon)
= 1 \}\]
 of $M^\times$ acts on $M$ by multiplication and is identified with a subgroup of $\operatorname{Aut}(M,Q)$; see \cite[Section 2.2]{NW76}. 

 The group $\mathfrak{A}$ has order $p^{\lambda-1}(p+1)$ and $\mathfrak{A} = \langle \alpha \rangle \times \langle \beta \rangle$. The relevant information for type $N$ quadratic modules is as follows: \begin{center}
\begin{tabular}{ccccc}$p$&
$\lambda$&
$\alpha$&
$\beta$&
$\mathfrak{A}_0$\\
\hline
$>2$&
$1$&
$(1,0)$&
$|\beta| = p+1$&
$\langle (1,0) \rangle$\\
$>2$&
$>1$&
$|\alpha| = p^{\lambda-1}$&
$|\beta| = p+1$&
$\langle \alpha \rangle$\\
$2$&
$1$&
$(1,0)$&
$|\beta| = 3$&
$\langle (1,0) \rangle$\\
$2$&
$2$&
$(1,0)$&
$|\beta| = 6$&
$\langle (-1,0) \rangle$\\
$2$&
$>2$&
$|\alpha| = p^{\lambda-2}$&
$|\beta| = 6$&
$\langle \alpha \rangle$\\
\end{tabular}\\[2mm]
\end{center}

 When $\mathfrak{A}_0$ is trivial, every non-trivial character $\chi \in \widehat{\mathfrak{A}}$ is primitive. 

 }

 
\subsection{\textcolor{Chapter }{Type R, generic cases}}\label{Chapter_Description_Section_Weil_Subsection_Type_R}
\logpage{[ 2, 2, 3 ]}
\hyperdef{L}{X7B788E2F87AF51CA}{}
{
  

 The structure of the quadratic module $(M,Q)$ of type $R$ depends upon three additional parameters: $\sigma$, $r$, and $t$. Details are as follows: 

 
\begin{itemize}
\item If $p$ is odd, let $\lambda \geq 2$, $\sigma \in \{1, \dots, \lambda\}$, and $r,t \in \{1,u\}$ with $u$ a quadratic non-residue mod $p$. Then define 
\[M = \mathbb{Z}/p^\lambda\mathbb{Z} \oplus
\mathbb{Z}/p^{\lambda-\sigma}\mathbb{Z} \qquad \text{and} \qquad Q(x,y) =
\frac{r(x^2 + p^\sigma t y^2)}{p^\lambda}~.\]
 When $\sigma = \lambda$, the second factor of $M$ is trivial, and $(M,Q)$ is said to be in the \emph{unary} family; otherwise, it is called \emph{generic}.
\item If $p=2$, let $\lambda \geq 2$, $\sigma \in \{0, \dots, \lambda-2\}$ and $r,t \in \{1,3,5,7\}$. Then define 
\[M = \mathbb{Z}/2^{\lambda-1}\mathbb{Z} \oplus
\mathbb{Z}/2^{\lambda-\sigma-1}\mathbb{Z} \qquad \text{and} \qquad Q(x,y) =
\frac{r(x^2 + 2^\sigma t y^2)}{2^\lambda}~.\]
 When $\sigma = \lambda - 2$, the second factor of $M$ is isomorphic to $\mathbb{Z}/2\mathbb{Z}$, and $(M,Q)$ is said to be in the \emph{extremal} family; otherwise, it is called \emph{generic}.
\end{itemize}
 

 In all cases, the resulting representation is said to be of type $R$ and denoted $R(p,\lambda,\sigma,r,t)$. The additive group $M$ admits a ring structure with multiplication 
\[(x_1, y_1) \cdot (x_2, y_2) = (x_1x_2 - p^\sigma ty_1y_2, x_1y_2 + x_2y_1)\]
 and identity element $(1,0)$. We define a norm $\operatorname{Nm}(x,y) = x^2 + xy + p^\sigma t y^2$ on $M$. 

 In this section, we detail generic type $R$ quadratic module. Information on the unary and extremal cases is covered in
Section \ref{Chapter_Description_Section_Weil_Subsection_Type_R_Special}. 

 Let $(M,Q)$ be a generic type $R$ quadratic modules. Information on $(M,Q)$ can be obtained via \texttt{SL2ModuleR} (\ref{SL2ModuleR}), and subrepresentations of $R(p,\lambda,\sigma,r,t)$ with level $p^\lambda$ may be constructed via \texttt{SL2IrrepR} (\ref{SL2IrrepR}). 

 The multiplicative subgroup 
\[\mathfrak{A} = \{\varepsilon \in M^\times \mid \operatorname{Nm}(\varepsilon)
= 1 \}\]
 of $M^\times$ acts on $M$ by multiplication and is identified with a subgroup of $\operatorname{Aut}(M,Q)$; see \cite[Section 2.3 - 2.4]{NW76}. The relevant information is as follows: 
\begin{itemize}
\item If $p$ is odd, $\mathfrak{A} = \langle\alpha\rangle \times \langle\beta\rangle$ with order $2p^{\lambda-\sigma}$. In this case, for fixed $p$, $\lambda$, $\sigma$, each pair $(r,t)$ gives rise to a distinct quadratic module \cite[Satz 4]{Nobs1}. The following table covers a complete list of representatives of equivalence
classes of such modules. \begin{center}
\begin{tabular}{ccccccc}$p$&
$\lambda$&
$\sigma$&
$(r,t)$&
$\alpha$&
$\beta$&
$\mathfrak{A}_0$\\
\hline
$3$&
$2$&
$1$&
$r,t \in \{1,2\}$&
$|\alpha| = 3$&
$(-1,0)$&
$\langle \alpha \rangle$\\
$3$&
$\geq 3$&
$1$&
$t=1$, $r \in \{1,2\}$&
$|\alpha| = 3^{\lambda-\sigma-1}$&
$|\beta| = 6$&
$\langle \alpha \rangle$\\
$3$&
$\geq 3$&
$1$&
$t=2$, $r \in \{1,2\}$&
$|\alpha| = 3^{\lambda-\sigma}$&
$(-1,0)$&
$\langle \alpha \rangle$\\
$3$&
$\geq 3$&
$2,\dots,\lambda-1$&
$r,t \in \{1,2\}$&
$|\alpha| = 3^{\lambda-\sigma}$&
$(-1,0)$&
$\langle \alpha \rangle$\\
$\geq 5$&
$\geq 2$&
$1, \dots,\lambda - 1$&
$r,t \in \{1,u\}$&
$|\alpha| = p^{\lambda-\sigma}$&
$(-1,0)$&
$\langle \alpha \rangle$\\
\end{tabular}\\[2mm]
\end{center}


\item If $p=2$, then the generic case occurs when $\lambda \geq 3$ and $\sigma \in \{0,\dots,\lambda-3\}$. Again, $\mathfrak{A} = \langle\alpha\rangle \times \langle\beta\rangle$; the order is $2^{\lambda-\sigma-k}$ with $k \in \{0,1,2\}$ (as specified below). In this case, for fixed $p$, $\lambda$, $\sigma$, two pairs $(r_1,t_1)$ and $(r_2,t_2)$ may give rise to equivalent quadratic modules \cite[Satz 4]{Nobs1}. The following table covers a complete list of representatives of equivalence
classes of such modules. \begin{center}
\begin{tabular}{ccccccc}$\lambda$&
$\sigma$&
$r$&
$t$&
$\alpha = (x,y)$&
$\beta$&
$\mathfrak{A}_0$\\
\hline
$3$&
$0$&
$1,3$&
$1,5$&
$(1,0)$&
$(\frac{t-1}{2},1)$&
$\langle (-1,0) \rangle$\\
$3$&
$0$&
$1$&
$3,7$&
$(1,0)$&
$(-1,0)$&
$\langle (-1,0) \rangle$\\
$4$&
$0$&
$1,3$&
$5$&
$x=2, y \equiv 3 \operatorname{mod} 4, |\alpha| = 2^{\lambda-2}$&
$(-1,0)$&
$\langle -\alpha^2 \rangle$\\
$\geq 4$&
$0$&
$1,3$&
$1$&
$x \equiv 1 \operatorname{mod} 4, y = 4, |\alpha| = 2^{\lambda-3}$&
$(0,1)$&
$\langle \alpha \rangle$\\
$\geq 4$&
$0$&
$1$&
$3,7$&
$x \equiv 1 \operatorname{mod} 4, y = 4, |\alpha| = 2^{\lambda-3}$&
$(-1,0)$&
$\langle \alpha \rangle$\\
$\geq 5$&
$0$&
$1,3$&
$5$&
$x=2, y \equiv 3 \operatorname{mod} 4, |\alpha| = 2^{\lambda-2}$&
$(-1,0)$&
$\langle \alpha \rangle$\\
$\geq 3$&
$1,2$&
$1,3,5,7$&
$1,3,5,7$&
$x\equiv 1 \operatorname{mod} 4, y=2, |\alpha| = 2^{\lambda-\sigma-2}$&
$(-1,0)$&
$\langle \alpha \rangle$\\
$\geq 3$&
$\geq 3$&
$1,3,5,7$&
$1,3,5,7$&
$x\equiv 1 \operatorname{mod} 4, y=1, |\alpha| = 2^{\lambda-\sigma-1}$&
$(-1,0)$&
$\langle \alpha \rangle$\\
\end{tabular}\\[2mm]
\end{center}


\end{itemize}
 

 }

 
\subsection{\textcolor{Chapter }{Type R, unary and extremal cases}}\label{Chapter_Description_Section_Weil_Subsection_Type_R_Special}
\logpage{[ 2, 2, 4 ]}
\hyperdef{L}{X7B1E74AC85CBD40B}{}
{
  

 This section covers the unary and extremal cases of type $R$. 

 First, in the unary family, we have $p$ odd and $\sigma = \lambda$. Then the second factor of $M$ is trivial (and hence $t$ is irrelevant). We then denote $R_{p^\lambda}(r) = R(p,\lambda,\lambda,r,t)$. In this case, we do not decompose $W(M,Q)$ using characters: instead, if $\lambda \leq 2$, then $W(M,Q)$ contains two distinct irreducible subrepresentations of level $p^\lambda$, denoted $R_{p^\lambda}(r)_{\pm}$; otherwise, it contains a single such subrepresentation, denoted $R_{p^\lambda}(r)_1$. The unary family is handled by \texttt{SL2IrrepRUnary} (\ref{SL2IrrepRUnary}) (which is called by \texttt{SL2IrrepR} (\ref{SL2IrrepR}) when appropriate). 

 Second, in the extremal family, we have $p=2$, $\lambda \geq 2$, and $\sigma = \lambda - 2$. Then the second factor of $M$ is isomorphic to $\mathbb{Z}/2\mathbb{Z}$, and collapses in $2M$. Here, $\operatorname{Aut}(M,Q)$ is itself abelian, so we let $\mathfrak{A} = \operatorname{Aut}(M,Q)$. This group has order 1, 2, or 4, with the following structure: 
\begin{itemize}
\item For $\lambda = 2$ and $t=1$, $\mathfrak{A} = \langle \tau \rangle$ where $\tau : (x,y) \mapsto (y,x)$, and $\mathfrak{A}_0 = \mathfrak{A} = \langle\tau\rangle$.
\item For $\lambda = 2$ and $t = 3$, $\mathfrak{A}$ is trivial; there are no primitive characters.
\item For $\lambda = 3$ or $4$, $\mathfrak{A} = \{\pm 1\}$ acting on $M$ by multiplication; there are no primitive characters.
\item Finally, for $\lambda \geq 5$, $\mathfrak{A} = \operatorname{Aut}(M,Q) = \langle \alpha \rangle \times \langle
-1 \rangle$ with $\alpha$ of order 2, and $\mathfrak{A}_0 = \langle\alpha\rangle$. Note that, for this special case, the usual test for primitivity (described
in Section \ref{Chapter_Description_Section_Construction}) fails, as there are no elements of $\mathfrak{A}$ fixing $2M$ pointwise.
\end{itemize}
 The extremal family is handled by \texttt{SL2ModuleR} (\ref{SL2ModuleR}) and \texttt{SL2IrrepR} (\ref{SL2IrrepR}), just like the generic case. 

 }

 }

 }

   
\chapter{\textcolor{Chapter }{Irreducible representations of prime-power level}}\label{Chapter_Irreps}
\logpage{[ 3, 0, 0 ]}
\hyperdef{L}{X7C4165447B8AB223}{}
{
  

 Methods for generating individual irreducible representations of $\mathrm{SL}_2(\mathbb{Z}/p^\lambda\mathbb{Z})$ for a given level $p^\lambda$. 

 In each case (except the unary type $R$, for which see \texttt{SL2IrrepRUnary} (\ref{SL2IrrepRUnary})), the underlying module $M$ is of rank $2$, so its elements have the form $(x,y)$ and are thus represented by lists \texttt{[x,y]}. 

 Characters of the abelian group $\mathfrak{A} = \langle\alpha\rangle \times \langle\beta\rangle$, have the form $\chi_{i,j}$, given by 
\[\chi_{i,j}(\alpha^{v}\beta^{w}) \mapsto
\mathbf{e}\left(\frac{vi}{|\alpha|}\right)
\mathbf{e}\left(\frac{wj}{|\beta|}\right)~,\]
 where $i$ and $j$ are integers. We therefore represent each character by a list \texttt{[i,j]}. Note that in some cases $\alpha$ or $\beta$ is trivial, and the corresponding index $i$ or $j$ is therefore irrelevant. 

 We write \texttt{p=}$p$, \texttt{lambda=}$\lambda$, \texttt{sigma=}$\sigma$, and \texttt{chi=}$\chi$. 
\section{\textcolor{Chapter }{Representations of type D}}\label{Chapter_Irreps_Section_Representations_of_type_D}
\logpage{[ 3, 1, 0 ]}
\hyperdef{L}{X82DA126D7F755B1C}{}
{
  

 See Section \ref{Chapter_Description_Section_Weil_Subsection_Type_D}. 

\subsection{\textcolor{Chapter }{SL2ModuleD}}
\logpage{[ 3, 1, 1 ]}\nobreak
\hyperdef{L}{X845D92CB7841CB0B}{}
{\noindent\textcolor{FuncColor}{$\triangleright$\enspace\texttt{SL2ModuleD({\mdseries\slshape p, lambda})\index{SL2ModuleD@\texttt{SL2ModuleD}}
\label{SL2ModuleD}
}\hfill{\scriptsize (function)}}\\
\textbf{\indent Returns:\ }
a record \texttt{rec(Agrp, Bp, Char, IsPrim)} describing $(M,Q)$. 



 Constructs information about the underlying quadratic module $(M,Q)$ of type $D$, for $p$ a prime and $\lambda \geq 1$. 

 \texttt{Agrp} is a list describing the elements of $\mathfrak{A}$. Each element $a \in \mathfrak{A}$ is represented in \texttt{Agrp} by a list \texttt{[v, a, a{\textunderscore}inv]}, where \texttt{v} is a list defined by $a = \alpha^{\mathtt{v[1]}} \beta^{\mathtt{v[2]}}$. Note that $\beta$ is trivial, and hence \texttt{v[2]} is irrelevant, when $\mathfrak{A}$ is cyclic. 

 \texttt{Bp} is a list of representatives for the $\mathfrak{A}$-orbits on $M^\times$, which correspond to a basis for the $\mathrm{SL}_2(\mathbb{Z}/p^\lambda\mathbb{Z})$-invariant subspace associated to any primitive character $\chi \in \widehat{\mathfrak{A}}$ with $\chi^2 \not\equiv 1$. For other characters, we must use different bases which are particular to
each case. 

 \texttt{Char(i,j)} converts two integers $i$, $j$ to a function representing the character $\chi_{i,j} \in \widehat{\mathfrak{A}}$. 

 \texttt{IsPrim(chi)} tests whether the output of \texttt{Char(i,j)} represents a primitive character. }

 

\subsection{\textcolor{Chapter }{SL2IrrepD}}
\logpage{[ 3, 1, 2 ]}\nobreak
\hyperdef{L}{X7FDB517981A2C091}{}
{\noindent\textcolor{FuncColor}{$\triangleright$\enspace\texttt{SL2IrrepD({\mdseries\slshape p, lambda, chi{\textunderscore}index})\index{SL2IrrepD@\texttt{SL2IrrepD}}
\label{SL2IrrepD}
}\hfill{\scriptsize (function)}}\\
\textbf{\indent Returns:\ }
a list of lists of the form $[S,T]$. 



 Constructs the modular data for the irreducible representation(s) of type $D$ with level $p^\lambda$, for $p$ a prime and $\lambda \geq 1$, corresponding to the character $\chi$ indexed by \texttt{chi{\textunderscore}index = [i,j]} (see the discussion of \texttt{Char(i,j)} in \texttt{SL2ModuleD} (\ref{SL2ModuleD})). 

 Depending on the parameters, $W(M,Q)$ will contain either 1 or 2 such irreps. }

 }

 
\section{\textcolor{Chapter }{Representations of type N}}\label{Chapter_Irreps_Section_Representations_of_type_N}
\logpage{[ 3, 2, 0 ]}
\hyperdef{L}{X7CD74CFC84B38042}{}
{
  

 See Section \ref{Chapter_Description_Section_Weil_Subsection_Type_N}. 

\subsection{\textcolor{Chapter }{SL2ModuleN}}
\logpage{[ 3, 2, 1 ]}\nobreak
\hyperdef{L}{X7A50CC5A7933E207}{}
{\noindent\textcolor{FuncColor}{$\triangleright$\enspace\texttt{SL2ModuleN({\mdseries\slshape p, lambda})\index{SL2ModuleN@\texttt{SL2ModuleN}}
\label{SL2ModuleN}
}\hfill{\scriptsize (function)}}\\
\textbf{\indent Returns:\ }
a record \texttt{rec(Agrp, Bp, Char, IsPrim, Nm, Prod)} describing $(M,Q)$. 



 Constructs information about the underlying quadratic module $(M,Q)$ of type $N$, for $p$ a prime and $\lambda \geq 1$. 

 \texttt{Agrp} is a list describing the elements of $\mathfrak{A}$. Each element $a \in \mathfrak{A}$ is represented in \texttt{Agrp} by a list \texttt{[v, a]}, where \texttt{v} is a list defined by $a = \alpha^{\mathtt{v[1]}} \beta^{\mathtt{v[2]}}$. Note that $\alpha$ is trivial, and hence \texttt{v[1]} is irrelevant, when $\mathfrak{A}$ is cyclic. 

 \texttt{Bp} is a list of representatives for the $\mathfrak{A}$-orbits on $M^\times$, which correspond to a basis for the $\mathrm{SL}_2(\mathbb{Z}/p^\lambda\mathbb{Z})$-invariant subspace associated to any primitive character $\chi \in \widehat{\mathfrak{A}}$ with $\chi^2 \not\equiv 1$. For other characters, we must use different bases which are particular to
each case. 

 \texttt{Char(i,j)} converts two integers $i$, $j$ to a function representing the character $\chi_{i,j} \in \widehat{\mathfrak{A}}$. 

 \texttt{IsPrim(chi)} tests whether the output of \texttt{Char(i,j)} represents a primitive character. 

 \texttt{Nm(a)} and \texttt{Prod(a,b)} are the norm and product functions on $M$, respectively. }

 

\subsection{\textcolor{Chapter }{SL2IrrepN}}
\logpage{[ 3, 2, 2 ]}\nobreak
\hyperdef{L}{X81D60FE878F02838}{}
{\noindent\textcolor{FuncColor}{$\triangleright$\enspace\texttt{SL2IrrepN({\mdseries\slshape p, lambda, chi{\textunderscore}index})\index{SL2IrrepN@\texttt{SL2IrrepN}}
\label{SL2IrrepN}
}\hfill{\scriptsize (function)}}\\
\textbf{\indent Returns:\ }
a list of lists of the form $[S,T]$. 



 Constructs the modular data for the irreducible representation(s) of type $N$ with level $p^\lambda$, for $p$ a prime and $\lambda \geq 1$, corresponding to the character $\chi$ indexed by \texttt{chi{\textunderscore}index = [i,j]} (see the discussion of \texttt{Char(i,j)} in \texttt{SL2ModuleN} (\ref{SL2ModuleN})). 

 Depending on the parameters, $W(M,Q)$ will contain either 1 or 2 such irreps. }

 }

 
\section{\textcolor{Chapter }{Representations of type R}}\label{Chapter_Irreps_Section_Representations_of_type_R}
\logpage{[ 3, 3, 0 ]}
\hyperdef{L}{X7D9759387C499FA0}{}
{
  

 See Section \ref{Chapter_Description_Section_Weil_Subsection_Type_R}. 

\subsection{\textcolor{Chapter }{SL2ModuleR}}
\logpage{[ 3, 3, 1 ]}\nobreak
\hyperdef{L}{X7B10D99E7AEAC411}{}
{\noindent\textcolor{FuncColor}{$\triangleright$\enspace\texttt{SL2ModuleR({\mdseries\slshape p, lambda, sigma, r, t})\index{SL2ModuleR@\texttt{SL2ModuleR}}
\label{SL2ModuleR}
}\hfill{\scriptsize (function)}}\\
\textbf{\indent Returns:\ }
a record \texttt{rec(Agrp, Bp, Char, IsPrim, Nm, Ord, Prod, c, tM)} describing $(M,Q)$. 



 Constructs information about the underlying quadratic module $(M,Q)$ of type $R$, for $p$ a prime. The additional parameters $\lambda$, $\sigma$, $r$, and $t$ should be integers chosen as follows. 

 If $p$ is an odd prime, let $\lambda \geq 2$, $\sigma \in \{1, \dots, \lambda - 1\}$, and $r,t \in \{1,u\}$ with $u$ a quadratic non-residue mod $p$. Note that $\sigma = \lambda$ is a valid choice for type $R$, however, this gives the unary case, and so is not handled by this function,
as it is decomposed in a different way; for this case, use \texttt{SL2IrrepRUnary} (\ref{SL2IrrepRUnary}) instead. 

 If $p=2$, let $\lambda \geq 2$, $\sigma \in \{0, \dots, \lambda-2\}$ and $r,t \in \{1,3,5,7\}$. 

 \texttt{Agrp} is a list describing the elements of $\mathfrak{A}$. Each element $a$ of $\mathfrak{A}$ is represented in \texttt{Agrp} by a list \texttt{[v, a]}, where \texttt{v} is a list defined by $a = \alpha^{\mathtt{v[1]}} \beta^{\mathtt{v[2]}}$. 

 \texttt{Bp} is a list of representatives for the $\mathfrak{A}$-orbits on $M^\times$, which correspond to a basis for the $\mathrm{SL}_2(\mathbb{Z}/p^\lambda\mathbb{Z})$-invariant subspace associated to any primitive character $\chi \in \widehat{\mathfrak{A}}$ with $\chi^2 \not\equiv 1$. For other characters, we must use different bases which are particular to
each case. 

 \texttt{Char(i,j)} converts two integers $i$, $j$ to a function representing the character $\chi_{i,j} \in \widehat{\mathfrak{A}}$. 

 \texttt{IsPrim(chi)} tests whether the output of \texttt{Char(i,j)} represents a primitive character. 

 \texttt{Nm(a)}, \texttt{Ord(a)}, and \texttt{Prod(a,b)} are the norm, order, and product functions on $M$, respectively. 

 \texttt{c} is a scalar used in calculating the $S$-matrix; namely $c = \frac{1}{|M|} \sum_{x \in M} \mathbf{e}(Q(x))$. Note that this is equal to $S_Q(-1) / \sqrt{|M|}$, where $S_Q(-1)$ is the central charge (see Section \ref{Chapter_Description_Section_Construction_Subsection_Weil_representations}). 

 \texttt{tM} is a list describing the elements of the group $M - pM$. }

 

\subsection{\textcolor{Chapter }{SL2IrrepR}}
\logpage{[ 3, 3, 2 ]}\nobreak
\hyperdef{L}{X80961A2C7C5F632E}{}
{\noindent\textcolor{FuncColor}{$\triangleright$\enspace\texttt{SL2IrrepR({\mdseries\slshape p, lambda, sigma, r, t, chi{\textunderscore}index})\index{SL2IrrepR@\texttt{SL2IrrepR}}
\label{SL2IrrepR}
}\hfill{\scriptsize (function)}}\\
\textbf{\indent Returns:\ }
a list of lists of the form $[S,T]$. 



 Constructs the modular data for the irreducible representation(s) of type $R$ with parameters $p$, $\lambda$, $\sigma$, $r$, and $t$, corresponding to the character $\chi$ indexed by \texttt{chi{\textunderscore}index = [i,j]} (see the discussions of $\sigma$, $r$, $t$, and \texttt{Char(i,j)} in \texttt{SL2ModuleN} (\ref{SL2ModuleN})). 

 Depending on the parameters, $W(M,Q)$ will contain either 1 or 2 such irreps. 

 If $\sigma = \lambda$ for $p \neq 2$, then the second factor of $M$ is trivial (and hence $t$ is irrelevant), so this falls through to \texttt{SL2IrrepRUnary} (\ref{SL2IrrepRUnary}). }

 

\subsection{\textcolor{Chapter }{SL2IrrepRUnary}}
\logpage{[ 3, 3, 3 ]}\nobreak
\hyperdef{L}{X7C94E3007A1BEE85}{}
{\noindent\textcolor{FuncColor}{$\triangleright$\enspace\texttt{SL2IrrepRUnary({\mdseries\slshape p, lambda, r})\index{SL2IrrepRUnary@\texttt{SL2IrrepRUnary}}
\label{SL2IrrepRUnary}
}\hfill{\scriptsize (function)}}\\
\textbf{\indent Returns:\ }
a list of lists of the form $[S,T]$. 



 Constructs the modular data for the irreducible representation(s) of unary
type $R$ (that is, the special case where $\sigma = \lambda$) with $p$ an odd prime, $\lambda$ a positive integer, and $r \in \{1,u\}$ with $u$ a quadratic non-residue mod $p$. 

 In this case, $W(M,Q)$ always contains exactly 2 such irreps. }

 }

 }

   
\chapter{\textcolor{Chapter }{Lists of representations}}\label{Chapter_Lists}
\logpage{[ 4, 0, 0 ]}
\hyperdef{L}{X8168FF017B2C0BB2}{}
{
  

 The \emph{degree} of a representation is also known as the \emph{dimension}. The \emph{level} of the congruent representation determined by the pair $(S,T)$ is equal to the order of $T$. 

 We assign to each representation a \emph{name} according to the conventions of \cite{NW76}. 
\section{\textcolor{Chapter }{Lists by degree}}\label{Chapter_Lists_Section_Degree}
\logpage{[ 4, 1, 0 ]}
\hyperdef{L}{X80406B597E11D7C6}{}
{
  

\subsection{\textcolor{Chapter }{SL2PrimePowerIrrepsOfDegree}}
\logpage{[ 4, 1, 1 ]}\nobreak
\hyperdef{L}{X7AFFB022821949A7}{}
{\noindent\textcolor{FuncColor}{$\triangleright$\enspace\texttt{SL2PrimePowerIrrepsOfDegree({\mdseries\slshape degree})\index{SL2PrimePowerIrrepsOfDegree@\texttt{SL2PrimePowerIrrepsOfDegree}}
\label{SL2PrimePowerIrrepsOfDegree}
}\hfill{\scriptsize (function)}}\\
\textbf{\indent Returns:\ }
a list of records of the form \texttt{rec(S, T, degree, level, name)}. 



 Constructs a list of all irreps of $\mathrm{SL}_2(\mathbb{Z})$ that have the given degree and prime power level. }

 

\subsection{\textcolor{Chapter }{SL2PrimePowerIrrepsOfDegreeAtMost}}
\logpage{[ 4, 1, 2 ]}\nobreak
\hyperdef{L}{X7CF672A47E035702}{}
{\noindent\textcolor{FuncColor}{$\triangleright$\enspace\texttt{SL2PrimePowerIrrepsOfDegreeAtMost({\mdseries\slshape maximum{\textunderscore}degree})\index{SL2PrimePowerIrrepsOfDegreeAtMost@\texttt{SL2PrimePowerIrrepsOfDegreeAtMost}}
\label{SL2PrimePowerIrrepsOfDegreeAtMost}
}\hfill{\scriptsize (function)}}\\
\textbf{\indent Returns:\ }
a list of records of the form \texttt{rec(S, T, degree, level, name)}. 



 Constructs a list of all irreps of $\mathrm{SL}_2(\mathbb{Z})$ that have at most the given maximum degree and prime power level. }

 

\subsection{\textcolor{Chapter }{SL2IrrepsOfDegree}}
\logpage{[ 4, 1, 3 ]}\nobreak
\hyperdef{L}{X7CC7D26D7E37A1B9}{}
{\noindent\textcolor{FuncColor}{$\triangleright$\enspace\texttt{SL2IrrepsOfDegree({\mdseries\slshape degree})\index{SL2IrrepsOfDegree@\texttt{SL2IrrepsOfDegree}}
\label{SL2IrrepsOfDegree}
}\hfill{\scriptsize (function)}}\\
\textbf{\indent Returns:\ }
a list of records of the form \texttt{rec(S, T, degree, level, name)}. 



 Constructs a list of all irreps of $\mathrm{SL}_2(\mathbb{Z})$ that have the given degree. }

 

\subsection{\textcolor{Chapter }{SL2IrrepsOfDegreeAtMost}}
\logpage{[ 4, 1, 4 ]}\nobreak
\hyperdef{L}{X78044A4683369ADC}{}
{\noindent\textcolor{FuncColor}{$\triangleright$\enspace\texttt{SL2IrrepsOfDegreeAtMost({\mdseries\slshape maximum{\textunderscore}degree})\index{SL2IrrepsOfDegreeAtMost@\texttt{SL2IrrepsOfDegreeAtMost}}
\label{SL2IrrepsOfDegreeAtMost}
}\hfill{\scriptsize (function)}}\\
\textbf{\indent Returns:\ }
a list of records of the form \texttt{rec(S, T, degree, level, name)}. 



 Constructs a list of all irreps of $\mathrm{SL}_2(\mathbb{Z})$ that have at most the given maximum degree. }

 }

 
\section{\textcolor{Chapter }{Lists by level}}\label{Chapter_Lists_Section_Level}
\logpage{[ 4, 2, 0 ]}
\hyperdef{L}{X78C583957D3FF6ED}{}
{
  

\subsection{\textcolor{Chapter }{SL2PrimePowerIrrepsOfLevel}}
\logpage{[ 4, 2, 1 ]}\nobreak
\hyperdef{L}{X7A0D2DB780D99B11}{}
{\noindent\textcolor{FuncColor}{$\triangleright$\enspace\texttt{SL2PrimePowerIrrepsOfLevel({\mdseries\slshape p, lambda})\index{SL2PrimePowerIrrepsOfLevel@\texttt{SL2PrimePowerIrrepsOfLevel}}
\label{SL2PrimePowerIrrepsOfLevel}
}\hfill{\scriptsize (function)}}\\
\textbf{\indent Returns:\ }
a list of records of the form \texttt{rec(S, T, degree, level, name)}. 



 Constructs a list of all irreps of $\mathrm{SL}_2(\mathbb{Z})$ with level exactly $p^\lambda$. }

 }

 
\section{\textcolor{Chapter }{Lists of exceptional representations}}\label{Chapter_Lists_Section_Exceptions}
\logpage{[ 4, 3, 0 ]}
\hyperdef{L}{X80400C6D79D4D0D6}{}
{
  

\subsection{\textcolor{Chapter }{SL2ExceptionalIrreps}}
\logpage{[ 4, 3, 1 ]}\nobreak
\hyperdef{L}{X85078197803C9848}{}
{\noindent\textcolor{FuncColor}{$\triangleright$\enspace\texttt{SL2ExceptionalIrreps({\mdseries\slshape arg})\index{SL2ExceptionalIrreps@\texttt{SL2ExceptionalIrreps}}
\label{SL2ExceptionalIrreps}
}\hfill{\scriptsize (function)}}\\
\textbf{\indent Returns:\ }
a list of records of the form \texttt{rec(S, T, degree, level, name)}. 



 Constructs a list of the 18 exceptional irreps of $\mathrm{SL}_2(\mathbb{Z})$. }

 }

 }

   
\chapter{\textcolor{Chapter }{Methods for testing}}\label{Chapter_Testing}
\logpage{[ 5, 0, 0 ]}
\hyperdef{L}{X794C1A137F8FA14D}{}
{
  
\section{\textcolor{Chapter }{Testing}}\label{Chapter_Testing_Section_Testing}
\logpage{[ 5, 1, 0 ]}
\hyperdef{L}{X7DE7E7187BE24368}{}
{
  

\subsection{\textcolor{Chapter }{SL2WithConjClasses}}
\logpage{[ 5, 1, 1 ]}\nobreak
\hyperdef{L}{X801680187A8CA462}{}
{\noindent\textcolor{FuncColor}{$\triangleright$\enspace\texttt{SL2WithConjClasses({\mdseries\slshape p, lambda})\index{SL2WithConjClasses@\texttt{SL2WithConjClasses}}
\label{SL2WithConjClasses}
}\hfill{\scriptsize (function)}}\\
\textbf{\indent Returns:\ }
the group $\mathrm{SL}_2(\mathbb{Z}/p^\lambda\mathbb{Z})$ with conjugacy classes set to the format we use. 



 

 }

 

\subsection{\textcolor{Chapter }{SL2ChiST}}
\logpage{[ 5, 1, 2 ]}\nobreak
\hyperdef{L}{X7A97C7C77C7608D0}{}
{\noindent\textcolor{FuncColor}{$\triangleright$\enspace\texttt{SL2ChiST({\mdseries\slshape S, T, p, lambda})\index{SL2ChiST@\texttt{SL2ChiST}}
\label{SL2ChiST}
}\hfill{\scriptsize (function)}}\\
\textbf{\indent Returns:\ }
a list representing a character of $\mathrm{SL}_2(\mathbb{Z}/p^\lambda\mathbb{Z})$. 



 Converts the modular data $(S,T)$, which must have level dividing $p^\lambda$, into a character of $\mathrm{SL}_2(\mathbb{Z}/p^\lambda\mathbb{Z})$, presented in a form matching the conjugacy classes used in \texttt{SL2WithConjClasses}. }

 

\subsection{\textcolor{Chapter }{SL2IrrepPositionTest}}
\logpage{[ 5, 1, 3 ]}\nobreak
\hyperdef{L}{X7FB0262F7E0F1B95}{}
{\noindent\textcolor{FuncColor}{$\triangleright$\enspace\texttt{SL2IrrepPositionTest({\mdseries\slshape p, lambda})\index{SL2IrrepPositionTest@\texttt{SL2IrrepPositionTest}}
\label{SL2IrrepPositionTest}
}\hfill{\scriptsize (function)}}\\
\textbf{\indent Returns:\ }
a boolean. 



 Constructs and tests all irreps of level dividing $p^\lambda$ by checking their positions in \texttt{Irr(G)} (see \href{https://www.gap-system.org/Manuals/doc/ref/chap71.html#X873B3CC57E9A5492} {Section 71.8-2 of the GAP Manual}). }

 }

 }

 \def\bibname{References\logpage{[ "Bib", 0, 0 ]}
\hyperdef{L}{X7A6F98FD85F02BFE}{}
}

\bibliographystyle{alpha}
\bibliography{SL2Reps.bib}

\addcontentsline{toc}{chapter}{References}

\def\indexname{Index\logpage{[ "Ind", 0, 0 ]}
\hyperdef{L}{X83A0356F839C696F}{}
}

\cleardoublepage
\phantomsection
\addcontentsline{toc}{chapter}{Index}


\printindex

\immediate\write\pagenrlog{["Ind", 0, 0], \arabic{page},}
\immediate\write\pagenrlog{["Ind", 0, 0], \arabic{page},}
\immediate\write\pagenrlog{["Ind", 0, 0], \arabic{page},}
\newpage
\immediate\write\pagenrlog{["End"], \arabic{page}];}
\immediate\closeout\pagenrlog
\end{document}
